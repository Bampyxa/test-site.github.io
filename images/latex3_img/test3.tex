\documentclass[a4paper,12pt]{article}

%\usepackage[T2A]{fontenc}
\usepackage[utf8]{inputenc}
\usepackage[english,russian]{babel}

% Работа с графикой
\usepackage{tikz}
\usepackage{pgfplots}
\usepackage{pgfplotstable}

\usetikzlibrary{calc}
\usetikzlibrary{positioning}%для блок-схем
\usetikzlibrary{shapes,shadows}%для блок-схем

\usepackage{fancyvrb}%"короткий verbatim"
\DefineShortVerb{\|}

%\tikzset{help lines/.style={ultra thin, color=blue}}%здесь задаются глоб. стили

\title{Графики и простые рисунки в \LaTeX}
\author{Какойто Какойтов}
\date{\today}

\begin{document}

\maketitle

%\section{Общие сведения}

%Для создания  в системе \LaTeX рисунков, графиков, схем и тому подобного контента применяется один из известных пакетов - PGF и его расширение TikZ. Строго говоря этот пакет известен не только в \LaTeX-е, но и может быть подключен и к другим программам MATLAB, Geogebra и другие.

%Преимущество пакета pgf (что означает portable graphics format — переносимый графический формат) в том, что не надо таскать в папке сопутствующие картинки, нужного размера и качества(конечно кроме фотографий), все нужные изображения будут находится в .tex файле в виде кода. Так же качество при отображении на экранах любых устройств будет всегда высоким.

%В Windows если у вас установлена полная версия \LaTeX(развёрнута из MikeTeX или LiveTeX) TIKZ и PGF будут идти в комплекте.



\section{Подготовка к работе}

Итак, запускаем любой \LaTeX редактор: TeXworks, TeXStudio и др. Создаём новый файл, а в нём:
\begin{itemize}
\item \verb|\documentclass[a4paper,12pt]{article}| - указываем класс и некоторые параметры документа(статья формата А4 и базовый шрифт - 12pt);
\item \verb|\usepackage[utf8]{inputenc}| - кодировка;
\item \verb|\usepackage[english,russian]{babel}| - поддержка русского языка;
\item \verb|\usepackage{tikz}| - включаем tikz;
\item \verb|\usepackage{pgfplots}| - включаем расширенный pgf;
\item \verb|\usepackage{pgfplotstable}| - то же;

\item \verb|\title{Название}| - параметры титульной страницы \\
\verb|\author{Автор}| \\
\verb|\date{Сегодня}| \\

\item \verb|\begin{document}| - начало документа \\
\verb|\maketitle| - выводим титульный лист \\
Здесь начинаем рисовать:) \\
\verb|\end{document}|
\end{itemize}



\section{Основы}
\subsection{Простая команда TIKZ}

Здесь используется простая команда \verb|\tikz{код}|:\\
\tikz{\draw (0, 0)--(1, 0)--(1, 1)--(0, 1)--(0, 0)}
\verb|\tikz{\draw (0, 0)--(1, 0)--(1, 1)--(0, 1)--(0, 0)}|\\
draw - рисовать что-то(линию, точку);\\
(0, 0) - координаты точки(x, y) - левый нижний угол;\\
(0, 0)--(1, 0) - линия между двумя точками;

\tikz[xscale=.5, yscale=2]{\draw (0, 0)--(1, 1)}
\verb|\tikz[xscale=.5, yscale=2]{\draw (0, 0)--(1, 1)}|\\
изменение размера по вертикали(xscale) и горизонтали(yscale);

\tikz{\draw [x=10pt,y=10pt] (0, 0)--(1, 2)}
\verb|\tikz{\draw[x=10pt,y=10pt] (0, 0)--(1, 2)}|\\
можно задать свои еденицы измерения(pt, px, sm, mm). По-умолчанию в \LaTeX используются свои ед. измерения.

%\tikz{\draw [help lines] (0, 0)--(1, 0)--(1, 1)--(0, 1)--(0, 0)} -вспомог. тонкие линии
%\tikz[ultra thick]{\draw [<->] (0, 0) -- (0, 2)} - толщина линий: ultra thin->very thin->thin->semithick->thick->very thick->ultra thick




\subsection{TIKZ-окружение}

Но эта короткая команда из одной строки используются для рисования простой фигуры или линии. Если нужно что-то посложнее, используется окружение tikzpicture. При этом точка отсчёта координат будет находится в начале строки и при рисовании различных линий и фигур надо это учитывать.\\

%прерывистость линий:
\begin{tikzpicture}%окружение
\draw [line width=2mm] (0, 0)--(2, 1);
\draw[dashed] (2, 1)--(4, 0);%свободной и плотной версии: loosely dashed, densely dashed, loosely dotted, densely dotted
\draw[loosely dashed] (4, 0)--(6, 1);
\draw[red] (6, 1)--(8, 0);
\draw[green](8, 0) -- ++(0, 0) -- +(1, 1) -- +(1, 0) -- +(2, 0);%изм-е начала коорд. относ-но старых значений
\end{tikzpicture}\\

\verb|\begin{tikzpicture}| - окружение рисунка\\
\verb|\draw [line width=2mm] (0, 0)--(2, 1);|
 - линия толщиной 2мм\\
\verb|\draw[dashed] (2, 1)--(4, 0);|
 - прерывистая линия\\
\verb|\draw[loosely dotted](4,0)--(6,1);|
 - разреженная прерывистая линия\\
\verb|\draw[red] (6, 1)--(8, 0);|
 - линия красного цвета\\
\verb|\draw[green](8, 0) -- ++(0, 0) -- +(1, 1) -- +(1, 0) -- +(2, 0);|
 - Обратите внимание, при рисовании этой линии строка: $(8, 0) -- ++(0, 0)$ обнулила точку начала координат и каждая новая точка будет откладываться от неё: $+(1, 1)$<<плюс обязателен>>\\
\verb|\end{tikzpicture}|\\

Толщина линий:\\
\begin{tikzpicture}[xscale=2]
\draw[ultra thin](0,0)--(0,1) node[below]{ultra thin};
\draw[very thin](1,0)--(1,1) node[below]{very thin};
\draw[thin](2,0)--(2,1) node[below]{thin};
\draw[semithick](3,0)--(3,1) node[below]{semithick};
\draw[thick](4,0)--(4,1) node[below]{thick};
\draw[very thick](5,0)--(5,1) node[below]{very thick};
\draw[ultra thick](6,0)--(6,1) node[below]{ultra thick};
\end{tikzpicture}

Прерывистость линий:\\%loosely dashed, densely dashed, loosely dotted, densely dotted
\begin{tikzpicture}[xscale=2]
\draw[loosely dashed](0,0)--(0,1) node[below]{loosely dashed};
\draw[densely dashed](1.5,0)--(1.5,1) node[below]{densely dashed};
\draw[loosely dotted](3,0)--(3,1) node[below]{loosely dotted};
\draw[densely dotted](4.5,0)--(4.5,1) node[below]{densely dotted};
\end{tikzpicture}



\subsection{Фигуры}
%фигуры:
\begin{tikzpicture}
%\draw [thin, blue] (0, 0) rectangle(2, 1.5);%прямоугольник
%\draw[dashed,green] (3, 1)  circle[radius=1];
%\draw[dashed,red] (3, 2)  circle(1.5);%сокращ-е
%\draw[purple] (6, 2) arc [radius=1, start angle=90, end angle=240];%дуга круга
%\draw[orange] (6, 0) arc(200:60:1.2);%сокращ-е
%\draw[ultra thick, ->] (8, 0) to [out=90, in=180] (9, 1) to [out=0, in=180] (11, 0);%кривая
%\draw[rounded corners=5] (12, 0)  rectangle(15, 1);%скругление углов
%\draw (0, 0) ellipse (2 and 1);%эллипс
%\draw (0,0) .. controls (1,1) and (2,1) .. (2,0);%кривая Безье
%\draw[blue](0,0) parabola (2,2);
%\draw [x=15.7pt,y=10pt] (2,0) sin (1,1) cos (2,0) sin (3,-1) cos (4,0) (0,1) cos (1,0) sin (2,-1) cos (3,0) sin (4,1);
\end{tikzpicture}\\

\tikz{\draw [thin, blue] (0, 0) rectangle(2, 1.5)} - прямоугольник\\
\verb|\draw [thin, blue] (0, 0) rectangle(2, 1.5);|\\
thin, blue - тонкая линия синего цвета;\\
(0,0) - точка отсчёта(левый нижний угол);\\
rectangle - прямоугольник;\\
(2,1.5) - координаты угла, противоположного точке начала коорд.\\

\tikz{\draw[dashed,green] (0, 0)  circle[radius=1]} - круг\\
\verb|\draw[dashed,green] (0, 0)  circle[radius=1];|\\
thin, blue - прерывистая линия зелёного цвета;\\
(3, 1) - точка отсчёта;\\
circle - круг;\\
radius=1.2 - радиус 1.2см, можно записать сокращённо: (1.2)\\
%\draw[dashed,red] (3, 2)  circle(1.5);%сокращ-е

\tikz{\draw[purple] (0, 0) arc [radius=1, start angle=90, end angle=240]} - дуга\\
\verb|\draw[purple] (0, 0) arc [radius=1, start angle=90,|\\
\verb|end angle=240];|\\
purple - фиолетовая линия;\\
(3, 1) - точка отсчёта;\\
arc - дуга;\\
radius=1 - с радиусом 1(с центром в воображаемой точке);\\
start angle=90 - точка начала дуги;\\
end angle=240 - точка конца дуги;\\
параметры в [] можно записать сокращённо - (240:90:1);\\
%\draw[orange] (6, 0) arc(200:60:1.2);%сокращ-е

\tikz{\draw[thick] (0,0) to [out=45,in=270] (3,2)} - кривая\\
\verb|\draw (0,0) to [out=45,in=270] (3,2)|\\
кривая толстая линия с началом в точке (0, 0);\\
выходит из неё под углом $45^o$\\
 и заходит в точку (3,2) под углом $270^o$\\
 
 \tikz{\draw (0, 0) ellipse (2 and 1);} - эллипс\\
 \verb|\draw (0, 0) ellipse (2 and 1);|\\
 эллипс с центром в точке (0, 0);\\
 вертикальный радиус - 2;\\
 горизонтальный радиус - 1;\\
 
 \tikz{\draw (0,0) .. controls (1,1) and (2,1) .. (2,0)} - кривая Безье\\
\verb|\draw (0,0) .. controls (1,1) and (2,1) .. (2,0)|\\
кривая Безье, выходящая из точки (0, 0)\\
и приходящая в точку (2, 0);\\
точки контроля кривой (1,1) и (2,1);



\subsection{Заливка фигур цветом}

\tikz{\draw[fill] (4, 1) circle[radius=.5]}\\
\verb|\draw[fill] (4, 1) circle[radius=.5];|\\
По умолчанию - заливка чёрным цветом\\

\tikz{\draw[fill=red, red] (0, 0) rectangle(2, 1)}\\
Красный прямоугольник\\
\verb|\draw[fill=red, red] (0, 0) rectangle(2, 1)|\\
fill=red - заливка красным цветом\\
ещё один red после - обводка фигуры красным же цветом

\begin{tikzpicture}
\draw[fill=red, red] (0, 0) rectangle(3, 2);
\draw[fill=blue, blue] (1, 1) rectangle(2, 2);
\end{tikzpicture}\\
Фигуры можно накладывать друг на друга(своя команда для каждой фигуры):
\begin{verbatim}
\begin{tikzpicture}
\draw[fill=red, red] (0, 0) rectangle(3, 2);
\draw[fill=blue, blue] (1, 1) rectangle(2, 2);
\end{tikzpicture}
\end{verbatim}

\tikz{\draw[fill=yellow] (0, 0)--(0, 1)--(1.5, 0)--(0, 0)}\\
Заливку и рамку фигуры можно указать такой командой...\\
\verb|\draw[fill=yellow] (0, 0)--(0, 1)--(1.5, 0)--(0, 0)|\\

\tikz{\filldraw[fill=magenta,draw=green,thick] (7, 0) to [out=90, in=180] (8, 1) --(8, 0)--(7,0)}\\
...или такой\\
\verb|\filldraw[fill=magenta,draw=green,thick] (7, 0) to [out=90, in=180] (8, 1) --(8, 0)--(7,0)|\\

\tikz{\path[fill=magenta] (9, 0)--(9, 1) to [out=0, in=90] (10, 0)--(9,0)}\\
\verb|\path[fill=magenta] (9, 0)--(9, 1) to [out=0, in=90] (10, 0)--(9,0);|\\
%\end{tikzpicture}



\subsection{Градиенты}

\tikz{\shade (0,0) rectangle (2,1)} - градиент-заливка\\
\verb|\shade (0,0) rectangle (2,1)|\\%градиент-заливка
По умолчанию переход от серого(gray) к белому(white)\\
\tikz{\shade[top color=cyan,bottom color=orange] (0,0) rectangle (2,1.5)}\\
\verb|\shade[top color=cyan,bottom color=orange] (0,0) rectangle (2,1.5)|\\
Градиент с указанием цветов(сверху вниз)\\
\tikz{\shadedraw[left color=green,right color=blue,draw=black, thick] (0,0) rectangle (2.5,1)}\\
\verb|\shadedraw[left color=green,right color=blue,draw=black, thick]|\\
\verb|(0,0) rectangle (2.5,1)|\\
Градиент справа налево с чёрной рамкой(-draw в названии команды указывает на наличие рамки)\\
\tikz{\shade[inner color=white,outer color=magenta] (0,0) rectangle +(2,2)}\\
\verb|\shade[inner color=white,outer color=magenta] (0,0) rectangle +(2,2)|\\
Радиальный градиент(внутри белый, снаружи малиновый)\\
\tikz{\shade[ball color=red] (14,.5) circle (.5cm)}\\
\verb|\shade[ball color=red] (14,.5) circle (.5cm)|\\
Предустановленный градиент в виде шара\\



\subsection{Коорд-я сетка и графики функций}
%координатная сетка и графики с подписями к ним:
\begin{tikzpicture}
\draw[help lines] (0, 0) grid (5, 5);
\draw[ultra thick, ->] (0, 0)--(0, 5) node[left] {$y$};
\draw[ultra thick, ->] (0, 0)--(5, 0) node[below] {$x$};
\draw[red, domain=0:2] plot (\x, \x*\x) node[right] {$y=x^2$};
\draw[blue, domain=0:3] plot (\x, \x) node[right] {$y=x$};
\draw[green, domain=0:4] plot (\x, {sin(\x*2 r)}) node[right] {$y=sinx*2$};
\end{tikzpicture}

\verb|1. \draw[help lines] (0, 0) grid (5, 5)|\\
Рисует сетку с ячейкой в 1см(по умолчанию) тонкими линиями

\verb|2. \draw[ultra thick, ->] (0, 0)--(0, 5) node[left] {$y$}|\\
Ось Y в виде стрелки, где:\\
\verb|node[left] {$y$}| - узел наименования(подпись) в виде буквы у слева от стрелки

\verb|3. \draw[ultra thick, ->] (0, 0)--(5, 0) node[below] {$x$}|\\
Здесь тоже самое для оси X

\verb|4. \draw[red, domain=0:2] plot (\x, \x*\x) node[right] {$y=x^2$}|\\
Этой строкой непосредственно рисуем график:\\
domain=0:2 - в каких пределах оси X отобразить график\\
plot (\verb|\x, \x*\x|) - сам график, где: \verb|\x| - икс(неизвестное), \verb|\x*\x| - формула графика - $x^2$

\verb|5. \draw[blue, domain=0:3] plot (\x, \x) node[right] {$y=x$}|\\
График попроще

\verb|6. \draw[green, domain=0:4] plot (\x, {sin(\x*2 r)}) node[right] {$y=sinx*2$}|\\
Здесь выражение функции обёрнуто в скобки \verb|{}|



\subsection{Координатные точки}

Как правило если нам нужна простая линия, кривая или фигура, можно её рисовать как мы делали выше. Если же рисунок сложнее и через одну точку проходит несколько линий лучше задать её с помощью специальной команды и обращаться к ней по заданному имени. Например:\\
\begin{tikzpicture}
\draw[help lines] (0, 0) grid (5, 5);
\draw[ultra thick, <->] (5, 0)--(0, 0)--(0, 5);
%координатные точки:
%\draw[fill] (1, 1) circle[radius=.1] node[below left] {$A$};
%\draw[fill] (2, 2) circle(.1) node[above right] {$B$};
%построение прямой по координатам:
\coordinate[label=below:$A$] (A) at(1,1);%использ-е координат
\coordinate[label=$B$] (B) at(2,3);
\coordinate[label=$C$] (C) at(3,2);%импровиз-е точки
\coordinate[label=right:$D$] (D) at(4,1);
\draw (A)--(B) (C)--(D);
\end{tikzpicture}

\verb|\coordinate[label=below:$A$] (A) at(1,1);|\\
coordinate - команда установки воображаемой точки по указанным далее координатам\\
label=below:\$A\$ - необязательный параметр, для отображения на рисунке названия данной точки(А) и некоторое её смещение вниз(below) относительно себя, чтобы было красивее:)\\
(А) - краткое название точки для дальнейшего её использования\\
at(1,1) - собственно сами координаты точки\\

Другие точки:
\begin{verbatim}
\coordinate[label=$B$] (B) at(2,3);
\coordinate[label=$C$] (C) at(3,2);
\coordinate[label=right:$D$] (D) at(4,1);
\end{verbatim}
\verb|\draw (A)--(B) (C)--(D);|\\
начертить оба отрезка можно одной командой, используя ранее данные имена точкам\\



\section{Простая диаграмма}
\subsection{Nodes в диаграмме}

\begin{tikzpicture}[node distance=3cm]
\node[fill=red!50] (node1) {Красный};
\node[shape=circle, fill=green!80!red, right of=node1] (node2) {Зелёный};
\node[fill=blue!50, below of=node2, node distance=1.5cm] (node4) {Синий};
\node[fill=orange!50, right of=node2] (node3) {Оранжевый};
\draw[->](node1) edge node[above] {t} (node2);
\draw[->](node3) edge node[above] {n} (node2);
\draw(node3) edge[->] (node4);
\draw(node1) edge[<-] (node4);
\end{tikzpicture}

\verb|1.\begin{tikzpicture}[node distance=3cm]|\\
окружение tikz с указанием для всех узлов(node) расстояния между ними(от центра до центра), для отдельных node расстояние можно указать внутри окружения

\verb|2.\node[fill=red!50] (node1) {Красный};|\\
node - создаём ноду,\\
fill= - заполняем её цветом(red!50 - означает берём 50\% красного и смешиваем с остальными 50\% по-умолчанию белого)\\
node1 - имя ноды для дальнейшего использования\\
\{...\} - здесь название ноды

\verb|3.\node[shape=circle, fill=green!80!red, right of=node1]|\\
\verb|(node2) {Зелёный};|\\
shape=circle - фигура ноды в виде круга(если не указано - прямоугольник)\\
fill=green!80!red - заполнение цветом(80\% зелёного и 20\% красного)\\
right of=node1 - указываем место, где нода будет изображена(справа от первой, на расстояние, определённое ранее - 3см)\\
затем имя и название ноды

\verb|4.\node[fill=blue!50, below of=node2, node distance=1.5cm]|\\
\verb|(node4)|\\
{Синий}; - здесь определяем расстояние самостоятельно\\
\verb|5.\node[fill=orange!50, right of=node2] (node3) {Оранжевый};|

\verb|6.\draw[->](node1) edge node[above] {t} (node2);|\\
рисуем между нодами(node1 и  node2) стрелку\\
edge - показывает как рисовать её(от края одной, до края другой ноды)\\
node[above] \{t\} - определяет ещё одну вспомогательную ноду, надпись над(above) стрелкой в виде буквы t

\verb|7.\draw[->](node3) edge node[above] {n} (node2);|

\verb|8.\draw(node3) edge[->] (node4);|\\
здесь между нодами вспомогательной ноды нет, пишем команду попроще

\verb|9.\draw(node1) edge[<-] (node4);|



\subsection{Разное}

\begin{tikzpicture}
\draw[help lines, step=.5cm] (-3,-2) grid (7, 2);
\draw[->] (-3.5,0)--(7.5,0);
\draw[->] (0,-2.5)--(0,2.5);
\draw[rotate=10] (-1, 0) ellipse (2 and 1) [rotate=30] (0, 0) ellipse(1. 5 and .5);
\draw (2, 0) circle(.8);
\draw (4, 1) arc(90:180:1);
\draw[rounded corners=5] (5, 1) rectangle(6.7, 1.7);
\fill[yellow!70!red] (5, 0) rectangle(6, .5);
\draw[fill=yellow!70!red, draw=green] (5, -.5) rectangle(6, -1);
\end{tikzpicture}

\verb|\draw[help lines, step=.5cm] (-3,-2) grid (7, 2);|\\
сетка со своим шагом, диапазоном координат и системой отсчёта

\verb|\draw[rotate=10] (-1,0) ellipse (2 and 1) [rotate=30] (0,0)|\\ \verb|ellipse(1. 5 and .5);|\\
два эллипса нарисованные одной командой с поворотом вокруг своей оси

\verb|\draw (2, 0) circle(.8)|\\
создание круга, краткая запись

\verb|\draw (4, 1) arc(90:180:1);|\\
создание дуги круга, краткая запись

\verb|\draw[rounded corners=5] (5, 1) rectangle(6.7, 1.7);|\\
прямоугольник со скруглёнными углами

\verb|\fill[yellow!70!red] (5, 0) rectangle(6, .5);|\\
создание прямоугольника, закрашенного смешанным цветом

\verb|\draw[fill=yellow!70!red, draw=green] (5,-.5) rectangle(6,-1)|\\
прямоугольник, нарисованный зелёной линией и залитый смешанным цветом


%Эллипс как рамка(обрезка):\\
%эллипс как рамка(обрезка)
\begin{tikzpicture}
\draw[clip] (0, 0) ellipse(2.5 and 1);
\draw[fill=yellow] (-1, -1) circle[radius=.5];
\draw[fill=red] (0, 0) rectangle(2, 1);
\draw (-2,0)--(-1,1)--(0,0)--(1,1)--(2,0);
\end{tikzpicture}\\
\verb|\draw[clip] (0, 0) ellipse(2.5 and 1);|\\
Здесь с помощью команды clip эллипс выступает в роли рамки, внутри которой видны другие фигуры, а снаружи нет\\

%Смещение фигуры относит. оси координат\\
\begin{tikzpicture}
\draw grid(2,2);
\draw[help lines, step=.5] grid(2,2);
\draw[shift={(.2,.2)}] (0,0) rectangle(1,1);
\end{tikzpicture}\\
\verb|\draw[shift={(.2,.2)}] (0,0) rectangle(1,1);|\\
Смещение фигуры на 0.2 еденицы относительно своего места.\\

Как правило смещение используется в циклах для рисования чего-либо через равные промежутки, например использование цикла для маркировки луча:\\
\begin{tikzpicture}
\draw[-latex] (-5,0)--(5,0);
\foreach \x in {-4,-3,-2,-1,0,1,2,3,4,}
\draw[shift={(\x,0)}] (0,.07)--(0,-.07) node[below]{$\x$};
\end{tikzpicture}
\begin{verbatim}
\draw[-latex] (-5,0)--(5,0);
\foreach \x in {-4,-3,-2,-1,0,1,2,3,4,}
\draw[shift={(\x,0)}] (0,.07)--(0,-.07) node[below]{$\x$};
\end{verbatim}



%Также предопределенные цвета можно смешивать. Синтаксис этой операции таков:
%color1!percent!color2
%В результате получится цвет, содержащий percent\% первого цвета и (100 - percent)\% второго. Если второй не указан — по умолчанию используется белый цвет. Смешивание цветов можно повторять, не сохраняя промежуточный цвет, например
%red!50!black!50!white
%Сначала будет получен цвет из 50\% красного и 50\% черного, а потом результат смешан с 50\% белого. В результирующем красного и черного цветов окажется по 25\%: 
%\draw[fill=red!50!yellow!50!green] rectangle(2,1);
%определение своего цвета:
%\definecolor{maincolor}{rgb}{0.5,0.1,0.7}
%\definecolor{other}{cmyk}{1,0.66,0.1,0.02} 



\section{Различные рисунки}
\subsection{Графики функций}

Система координат и функции.\\
\begin{tikzpicture}%сист. координат и ф-и
\draw[-latex] (-4,0)--(4,0);
\foreach \x in {-3,-2,-1,1,2,3}
\draw[shift={(\x,0)}] (0,.07)--(0,-.07) node[below]{$\x$};
\draw[-latex] (0,-4)--(0,4);
\foreach \y in {-3,-2,-1,1,2,3}
\draw[shift={(0,\y)}] (.07,0)--(-.07,0) node[left]{$\y$};
\draw (0,-.33) node[right] {$0$};
\draw[thick,domain=-3:1] plot(\x, -\x-2);
\draw[thick,domain=-1:3] plot(\x, 2-\x);
\draw[thick,domain=-1.6:1.6] plot(\x, -\x^3);
\end{tikzpicture}

Ещё функция\\
\begin{tikzpicture}%сист. координат и ф-и
\draw[-latex] (-5,0)--(1,0) node[below]{$x$};
\draw[-latex] (0,-2)--(0,2) node[left]{$y$};
\draw[thick,smooth,domain=-2.3:2.3] plot(-{\x*\x}, \x);%\sqrt{x}
%\draw[thick,domain=-4:4] plot(\x, {1/\x});
% \draw[dashed,domain=-4:4] plot(\x, 3+1/\x^2);
\end{tikzpicture}


\subsection{Отрезки и плоские фигуры}

Нахождение точек на прямой:\\
%нахождение точек на прямой
\verb|\coordinate[label=below:$A$] (A) at(0,0);|\\
\verb|\coordinate[label=below:$B$] (B) at(3,2);|\\
- определяем координатные точки A и B на рисунке: [название точки на рисунке] (сокращенное название для последующего использования в командах) at(x, y)-координаты точки\\
\verb|\draw (A)--(B);|\\
- прочертим между ними линию\\
\verb|\draw (A)  ($(A)!.5!(B)$) coordinate [label=below:$K$] (K);|\\
- нахождение точки K на прямой АB: AK =1/2АB(не чертя лишнюю линию)\\
\verb|\draw (A)  ($(A)!.3!(K)$) coordinate [label=below:$M$] (M);|\\
- нахождение точки M: AM = 1/3AK\\
\verb|\fill (K) circle(2pt) (M) circle(2pt);|\\
- обозначим найденные точки черными кружками\\
\begin{tikzpicture}
\coordinate[label=below:$A$] (A) at(0,0);
\coordinate[label=below:$B$] (B) at(3,2);
\draw (A)--(B);
\draw (A)  ($(A)!.5!(B)$) coordinate [label=below:$K$] (K);
\draw (A)  ($(A)!.3!(K)$) coordinate [label=below:$M$] (M);
\fill (K) circle(2pt) (M) circle(2pt);
\end{tikzpicture}

%параллелограмм
\begin{tikzpicture}
\coordinate[label=below:$A$] (A) at(0,0);%название точки А и смещение вниз
\coordinate[label=$B$] (B) at(1,3);
\coordinate[label=$C$] (C) at(6,3);
\coordinate[label=-90:$D$] (D) at(5,0);%название точки D и смещение вниз
\draw (A)--(B)--(C)--(D)--cycle;
\draw (A)--(C) (B)--(D);
\draw (A)  ($(A)!.5!(C)$) coordinate [label=below:$O$] (O);
\draw (A)  ($(A)!.5!(O)$) coordinate [label=below:$M$] (M);
\fill (O) circle(2pt) (M) circle(2pt);
%\fill (intersection of A--C and B--D) node[above]{O} circle (2pt);
%\node[fill=black,circle=2pt,label=above:$O$] (O) at ($(A)!.5!(C)$) {};
\end{tikzpicture}

%треугольник
\begin{tikzpicture}
\coordinate[label=-90:$A$] (A) at(-4,1);
\coordinate[label=90:$B$] (B) at(-2,4);
\coordinate[label=-90:$C$] (C) at(0,1);
\draw (A)--(B)--(C)--cycle;
\draw (B) -- ($(A)!.5!(C)$) coordinate [label=below:$K$] (K);
\fill (K) circle(2pt);
%\coordinate[label=-90:$K$] (K) at(-2,1);
%\draw (B)--(K);
%\fill (intersection of A--C and B--K) node[above]{O} circle (2pt);
\end{tikzpicture}

%прямоугольник
\begin{tikzpicture}
\draw[-latex] (-.5,0)--(6,0) node[below] {$x$};
\draw[-latex] (0,-.5)--(0,3) node[left] {$y$};
\coordinate[label=below left:$A$] (A) at(0,0);
\coordinate[label=left:$B$] (B) at(0,2);
\coordinate[label=right:$C$] (C) at(5,2);
\coordinate[label=-90:$D$] (D) at(5,0);
\coordinate[label=60:$M$] (M) at(2.5,3);
\draw (A)--(B)--(C)--(D)--cycle;
\draw[dashed] (A)--(M) (B)--(M) (C)--(M) (D)--(M);
\fill (M) circle(2pt);
\end{tikzpicture}

%трапеция
\begin{tikzpicture}
\coordinate[label=below:$A$] (A) at(0,0);
\coordinate[label=$B$] (B) at(1,3);
\coordinate[label=$C$] (C) at(4,3);
\coordinate[label=-90:$D$] (D) at(5,0);
\draw (A)  ($(A)!.5!(B)$) coordinate [label=left:$M$] (M);
\draw (C)  ($(C)!.5!(D)$) coordinate [label=right:$K$] (N);
\fill (M) circle(2pt) (N) circle(2pt);
%\coordinate[label=left:$M$] (M) at(.5,1.5);
%\coordinate[label=right:$N$] (N) at(4.5,1.5);
\draw (A)--(B)--(C)--(D)--cycle (M)--(N);
\end{tikzpicture}

%круг
\begin{tikzpicture}[scale=.5]
\draw[-latex] (-5,0)--(5,0) node[below] {$x$};
\draw[-latex] (0,-5)--(0,5) node[left] {$y$};
\coordinate[label=below left:$A$] (A) at(0,0);
\coordinate[label=$B$] (B) at(3,2.65);
\coordinate[label=above right:$C$] (C) at(4,0);
\coordinate[label=-90:$D$] (D) at(3,-2.65);
\draw (A)--(B)--(D)--cycle;
\fill[scale=2] (A) circle(2pt) (B) circle(2pt) (C) circle(2pt) (D) circle(2pt);
\draw (0,0) circle(4);
\end{tikzpicture}

%многоугольник
\begin{tikzpicture}
\coordinate[label=$A$] (A) at(-1.15,2);
\coordinate[label=$B$] (B) at(1.15,2);
\coordinate[label=right:$C$] (C) at(2.3,0);
\coordinate[label=below:$D$] (D) at(1.15,-2);
\coordinate[label=below:$E$] (E) at(-1.15,-2);
\coordinate[label=left:$F$] (F) at(-2.3,0);
\draw (A)--(B)--(C)--(D)--(E)--(F)--cycle;
\draw[dashed] (A)--(D) (B)--(E) (C)--(F);
\fill (intersection of A--D and B--E) node[above left]{$O$};
\draw (0,0) circle(2);
\end{tikzpicture}



\subsection{Объёмные фигуры}

%параллелепипед
\begin{tikzpicture}
\coordinate[label=below:$A$] (A) at(0,0);
\coordinate[label=left:$B$] (B) at(1,2);
\coordinate[label=right:$C$] (C) at(5,2);
\coordinate[label=-90:$D$] (D) at(4,0);
\coordinate[label=left:$A_1$] (A1) at(0,3);
\coordinate[label=$B_1$] (B1) at(1,5);
\coordinate[label=$C_1$] (C1) at(5,5);
\coordinate[label=right:$D_1$] (D1) at(4,3);
\draw (A)--(D)--(C) (A1)--(B1)--(C1)--(D1)--cycle (A)--(A1) (D)--(D1) (C)--(C1);
\draw[dashed] (A)--(B)--(C) (B)--(B1);
\end{tikzpicture}

%цилиндр
\begin{tikzpicture}
\coordinate[label=below:$A$] (A) at(0,0);
\coordinate[label=above:$B$] (B) at(0,3);
\coordinate[label=right:$C$] (C) at(2,3);
\coordinate[label=right:$D$] (D) at(2,0);
\draw (A)--(B)--(C)--(D)--cycle;
\draw (B) ellipse (2 and 1);
\draw (C) -- ($(C)!2!(B)$) coordinate [label=left:$M$] (M);
\draw (D) -- ($(D)!2!(A)$) coordinate [label=left:$N$] (N);
\draw (M)--(N);
\draw[dashed] (D) .. controls (2,1.3) and (-2,1.3) .. (N);
\draw (D) .. controls (2,-1.3) and (-2,-1.3) .. (N);
\end{tikzpicture}

\end{document}