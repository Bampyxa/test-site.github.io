\documentclass[a4paper,12pt]{article}%[leqno]-нумерация формул слева

%%% Работа с русским языком
\usepackage{cmap}		% поиск в PDF
\usepackage{mathtext} 	% русские буквы в формулах
\usepackage[T2A]{fontenc}	% кодировка
\usepackage[utf8]{inputenc}	% кодировка исходного текста(специально для русского языка: cp866 (DOS-кодировка), cp1251 (Windows-кодировка) или кодировка koi8-r)
\usepackage[english,russian]{babel}	%русификация служебных команд, названия математических ф-й, принятые в рус. традиции, соблюдение традиционного русского написания и др.
%\usepackage{indentfirst}%красная строка у 1-го и остальных абзацев(для рус.яз.)
\frenchspacing%тонкая настройка пробелов

%%% Дополнительная работа с математикой
\usepackage{amsmath,amsfonts,amssymb,amsthm,mathtools,amscd} % пакеты AMS
\usepackage{icomma} % "Умная" запятая: $0,2$ --- число, $0, 2$ --- перечисление
\usepackage[matrix,arrow,curve]{xy} %более сложные коммутатив. диагр.

%% Номера формул
%\mathtoolsset{showonlyrefs=true} % Показывать номера только у тех формул, на которые есть \eqref{} в тексте.
%\usepackage{leqno} % Нумереация формул слева

%% Шрифты
\usepackage{euscript} % Шрифт Евклид
\usepackage{mathrsfs} % Красивый матшрифт

% \renewcommand{\epsilon}{\ensuremath{\varepsilon}}%греч.буквы
% \renewcommand{\phi}{\ensuremath{\varphi}}
% \renewcommand{\kappa}{\ensuremath{\varkappa}}
% \renewcommand{\le}{\ensuremath{\leqslant}}%скобки
% \renewcommand{\leq}{\ensuremath{\leqslant}}
% \renewcommand{\ge}{\ensuremath{\geqslant}}
% \renewcommand{\geq}{\ensuremath{\geqslant}}
% \renewcommand{\emptyset}{\varnothing}%ничего

%% Свои команды
\DeclareMathOperator{\sgn}{\mathop{sgn}}

%% Перенос знаков в формулах (по Львовскому)
\newcommand*{\hm}[1]{#1\nobreak\discretionary{}
{\hbox{$\mathsurround=0pt #1$}}{}}

%%% Работа с картинками
\usepackage{graphicx} % Для вставки рисунков
\graphicspath{{images/}{images2/}} % папки с картинками
\setlength\fboxsep{3pt}% Отступ рамки \fbox{} от рисунка
\setlength\fboxrule{1pt}% Толщина линий рамки \fbox{}
\usepackage{wrapfig}% Обтекание рисунков и таблиц текстом

%%% Работа с таблицами
\usepackage{array,tabularx,tabulary,booktabs}% Дополнительная работа с таблицами
\usepackage{longtable}% Длинные таблицы
\usepackage{multirow} % Слияние строк в таблице
\usepackage{hhline}% настройка разделителей ячеек

%%% Теоремы
\theoremstyle{plain}%Это стиль по умолчанию, его можно не переопределять.
\newtheorem{theorem}{Теорема}[section]%подчинение счётч. разделов - 1.1
\newtheorem{proposition}[theorem]{Утверждение}%нумеруются вместе с теоремами - 1.2
 
\theoremstyle{definition}%Стиль "Определение"
\newtheorem{corollary}{Следствие}[theorem]%1.1.1
\newtheorem{problem}{Задача}[section]
 
\theoremstyle{remark}%Стиль "Примечание"
\newtheorem*{nonum}{Решение}

%%% Программирование
\usepackage{etoolbox} % логические операторы(true-false)

%%% Страница
%\usepackage{extsizes} % Возможность сделать 14-й шрифт
\usepackage{geometry} % Простой способ задавать поля
	\geometry{top=25mm}
	\geometry{bottom=35mm}
	\geometry{left=35mm}
	\geometry{right=20mm}
 %
\usepackage{fancyhdr} % Колонтитулы
 	%\pagestyle{fancy}
 	%\renewcommand{\headrulewidth}{1mm}  % Толщина линейки, отчеркивающей верхний колонтитул
 	% \lfoot{Нижний левый}
 	% \rfoot{Нижний правый}
 	% \rhead{Верхний правый}
 	% \chead{Верхний в центре}
 	% \lhead{Верхний левый \thepage из }
 	% \cfoot{Нижний в центре} % По умолчанию здесь номер страницы

\usepackage{setspace} % Интерлиньяж(междустрочный инт.)
%\onehalfspacing % Интерлиньяж 1.5
%\doublespacing % Интерлиньяж 2
%\singlespacing % Интерлиньяж 1

\usepackage{lastpage} % Узнать, сколько всего страниц в документе.

\usepackage{soulutf8} % Модификаторы начертания

%%Ссылки
\usepackage{hyperref}
\usepackage[usenames,dvipsnames,svgnames,table,rgb]{xcolor}
\hypersetup{			% Гиперссылки
    unicode=true,          % русские буквы в раздела PDF
    pdftitle={Заголовок},  % Заголовок
    pdfauthor={Автор},     % Автор
    pdfsubject={Тема},     % Тема
    pdfcreator={Создатель},% Создатель
    pdfproducer={Производитель},% Производитель
    pdfkeywords={keyword1} {key2} {key3},% Ключевые слова
    colorlinks=true,       % false: ссылки в рамках; true: цветные ссылки
    linkcolor=red,         % внутренние ссылки
    citecolor=green,       % на библиографию
    filecolor=magenta,     % на файлы
    urlcolor=cyan          % на URL
}

%\renewcommand{\familydefault}{\sfdefault} % Начертание шрифта(без засечек)

%%Работа с простой библиографией(не BIBLATEX)
\usepackage{cite}
\usepackage{csquotes}% Еще инструменты для ссылок(в т.ч. для BIBLATEX)

%%Работа с BIBLATEX
%\usepackage[backend=biber,bibencoding=utf8,sorting=ynt,maxcitenames=2,style=authoryear]{biblatex}
%\addbibresource{bib1.bib}
%\addbibresource{bib2.bib}

%\usepackage[superscript]{cite} % Ссылки в верхних индексах
%\usepackage[nocompress]{cite} % 

\usepackage{multicol}% Несколько колонок

%%Работа с графикой
\usepackage{tikz}
\usepackage{pgfplots}%графики
\usepackage{pgfplotstable}%
%\usepackage{mathrsfs}
%\usetikzlibrary{arrows}

\usepackage{soul}%выделение определений и др. с бОльшими интервалами между буквами: \so{определение}
\usepackage{color}%работа с цветом: \fcolorbox{black}{lgreen}{текст на зелёном с черной окантовкой}

%\renewcommand{\thezadacha}{\thesection.\arabic{zadacha}}




%%% Заголовок
\author{Anton}
\title{\LaTeX Тест документ}
\date{\today}%можно не указ.(подстав-ся сегодняшняя дата)

\begin{document} % конец преамбулы, начало документа

\maketitle%-вывод на экран заголовка документа
\begin{center}
Version 0.2
\end{center}



\section{Титульный лист}

\newpage% с новой стр.
\thispagestyle{empty}%очистить эту стр. от лишнего оформления

\begin{center}
\textit{Федеральное государственное автономное учреждение \\ высшего профессионального сказочного образования} \\
\vspace{0.5ex}
%
\textbf{НАЦИОНАЛЬНЫЙ ПОЛУ-ИССЛЕДОВАТЕЛЬСКИЙ УНИВЕРСИТЕТ \\ <<ВЫСШАЯ ШКОЛА ВСЕГОЗНАНИЯ>>}
\end{center}

\vspace{13ex}

\begin{flushright}%выравнивание по прав. краю
\noindent%убрать красную строку
\textit{Бессмертных Кащей Феоктистович} \\
%
\textit{студент факультета эзотерики \\ (группа 211Э)}
\end{flushright}

\begin{center}
\vspace{13ex}
\textbf{Р\,Е\,Ф\,Е\,Р\,А\,Т}\\
\vspace{1ex}
по алхимии \\
на тему: \\
\textbf{\textit{<<превращение железа в золото \\
с помощью философского камня>>}} \\
\vfill%заполнение верт. пробел. до след-го текста:
Москва 2015
\end{center}

\newpage


\section{Ошибки в \LaTeX}

Реакция на ошибки при трансляции:
\begin{itemize}
\item Enter-продолжить трансляцию до след. ошибки
\item x+Enter- прервать трансляцию
\item s+Enter- продолжить трансляцию до конца
\item i+Enter- ввести исправленную команду в консоль и продолжить трансляцию
\item h+Enter- справка об ошибке
\end{itemize}



\section{Начало документа и преамбула}
\subsection{Классы и настройки документа}

\verb|\documentclass[a4paper]{article}|- обязательная строка в преамбуле\\
Обязательный аргумент{} - класс документа:
\begin{itemize}
\item article удобно применять для статей
\item report — для более крупных статей, разбитых на главы, или небольших книг
\item book — для книг
\item beamer - презентации
\end{itemize}

\noindent Необязательные аргументы[](через запятую):
\begin{itemize}
\item a4paper- формат бумаги А4
\item landscape- альбомная ориентация
\item twoside- разные поля слева и справа(art-oneside)
\item twocolumn- 2 колонки(proc-всегда)
\item draft-черновой вариант с помет. длинных строк
\item fleqn/leqno- выключные формулы/номера располож. слева
\end{itemize}

\noindent Некоторые важные подключаемые модули:
\begin{itemize}
\item \verb|\usepackage[T2A]{fontenc}|- кодировка
\item \verb|\usepackage[utf8]{inputenc}|- кодировка исходного текста
\item \verb|\usepackage[english,russian]{babel}|- локализация и переносы
\item \verb|\usepackage{graphicx}|- Для вставки рисунков
\item \verb|\graphicspath{{images/}{images2/}}|- путь папки с картинками
\item \verb|\usepackage{hyperref}|- создание ссылок
\item \verb|\usepackage{tikz}|-продвинутая графика
\item \verb|\usepackage{pgfplots}|- создание графиков
\end{itemize}



\subsection{Иерархия названий}

\begin{itemize}
\item[] Part(часть)-
\item[] Chapter(часть-в report и book)-
\item[] Section(параграф)-
\item[] Subsection-
\item[] Subsubsection-
\item[] Paragraph(абзац)-
\item[] Subparagraph
\end{itemize}

\noindent\verb|\section{Название параграфа}|



\subsection{Общие ссылки}

\verb|\label{маркер}|- ставится у того объекта на который надо сослаться \\
\verb|\pageref{маркер}|- ссылка на страницу с этим маркером(ставится где угодно)\\
\verb|\ref{маркер}|- ссылка на:
\begin{itemize}
\item команды рубрикации текста (chapter, section,| и т.п.); те из них 
(наиболее «мелкие»), что номеров разделов не печатают, влияния 
на команду ref не оказывают; 
\item окружения, создающие нумерованные выключные формулы (та- 
кие, как equation и eqnarray, а также многочисленные окружения, 
определенные в пакете amsmath: multline, gather, align и иже с 
ними);
\item команда \verb|\caption| картинка;
\item команда \verb|\item| в окружении enumerate; 
\item команда \verb|\cite| источник цититирования из библиографического списка
\end{itemize}



\subsection{Стили}

\verb|\newpage| -разрыв страницы(новая стр.) \\
\verb|\clearpage| -то же, но если есть плавающие элем.(картинки, таблицы), будут напечатаны сначала они \\
\verb|\pagestyle{}|-стиль всех страниц:
\begin{itemize}
\item empty-нет колонтитулов и номеров стр.
\item plain-нет колонтитулов, стр. посередине внизу
\item headings-номер в колонтитуле
\item myheadings-то же со своими настройками
\end{itemize}
\verb|\thispagestyle{}|-стиль данной страницы \\
Ширина страницы = \verb|\textwidth=7cm|(ширина текста) + \verb|\oddsidemargin=0pt|(левое поле = 1 дюйм(default)+0pt) + правое поле-автоматически \\
\verb|\section[О слонах]{Кое-что о слонах}|-необязательный атрибут - для оглавления и колонтитулов \\
\verb|\titlepage|-(в преамб.) для создания титульного листа(в article)


\section{Текст}
\subsection{Строки}

Пробелы между словами:
\begin{itemize}
\item \verb|\!| отрицательный\!тонкий\!пробел
\item \verb|\,| тонкий\,пробел
\item \verb|\:| средний\:пробел
\item \verb|\;| толстый\;пробел
\item обычный пробел
\item \verb|\quad| пробел\quad1em
\item \verb|\qquad| пробел\qquad2em \\
\end{itemize}
Слово \raisebox{2pt}{подскочило} - \verb|\raisebox{2pt}{подскочило}| \\
в строке, а это \raisebox{-2pt}{провалилось} - \verb|\raisebox{-2pt}{провалилось}| \\
\verb|\centerline{Центрирует отдельную строку}| - \\
\centerline{Строка по центру} \\
\verb|Абракадабра\footnote{сноска внизу страницы}| --Абракадабра\footnote{неизвестное слово}--сказал он.

В строку\qquad 
\parbox{4cm}{вставили целый абзац текста, сверстанного по всем \TeX’овским правилам, шириной 4см. Можно также вставлять внутритекстовую высокую матаматическую формулу. После этого продолжается}
\qquad прерванная строка. \\
\begin{verbatim}
В строку\qquad 
\parbox{4cm}{вставили целый абзац текста, сверстанного по всем 
\TeX овским правилам, шириной 4см. Можно также вставлять внутритексто-
вую высокую матаматическую формулу. После этого продолжается}
\qquad прерванная строка.
\end{verbatim}
\fbox{текст или формула в рамке}, если нужно поместить сюда целый абзац, вставляем текст в окружение \verb|\parbox|
\begin{verbatim}
\fbox{текст или формула в рамке, если нужно поместить сюда целый 
абзац, вставляем текст в окружение \parbox}
\end{verbatim}




\subsection{Абзацы}

\verb|\par или \\[5pt]| конец абзаца(перевод строки), после можно 
указать необязательный аргумент, расстояние до след. строки \\
\verb|\smallskip|- задает стандартные промежутки между абзацами(\verb|\medskip, \bigskip|) \\
\verb|\vspace{2ex}/\hspace{1cm}| - вертикальный/горизонтальный промежуток \\
\verb|\noindent| или \verb|\parindent=0pt| - убирает нежелательный отступ в первой строке(красную строку) или делает его больше/меньше чем стандартный \\

\hrule\smallskip
Внутри абзаца тоже будет \vrule{} линейка.
\smallskip\hrule
\bigskip
\noindent \verb|\rule[-3pt]{5pt}{5pt}|- Настройки линейки: сдвиг влево-3pt, ширина и высота- 5pt \\
Как правило, для верт. линейки указывают ширину, а для горизонтальной - высоту: \\
\verb|\vrule width 2mm| \\
\verb|\hrule height 2mm| \\
\noindent\verb|\underline{подлежащее\strut}| - распорка или невидимая линейка, для того, чтобы разные слова были одинаковой фактической высоты, например для подчеркивания:
\underline{\strut слова}
\underline{люлька}
\underline{подряд\strut}



\subsection{Кегль - размеры шрифта}

\begin{table}[!h]
\caption{Размеры шрифта}
\centering
\begin{tabular}{|l|l|}
 \hline \verb|\tiny|         & {\tiny крошечный }\\
 \hline \verb|\scriptsize|   & {\scriptsize очень маленький} \\
 \hline \verb|\footnotesize| & {\footnotesize довольно маленький} \\
 \hline \verb|\small|        & {\small маленький }\\
 \hline \verb|\normalsize|   & {\normalsize нормальный }\\
 \hline \verb|\large|        & {\large большой }\\
 \hline \verb|\Large|        & {\Large еще больше} \\
 \hline \verb|\LARGE|        & {\LARGE очень большой} \\
 \hline \verb|\huge|         & {\huge огромный }\\
 \hline \verb|\Huge|         & {\Huge громадный }\\
 \hline
\end{tabular}
\end{table}

\noindent\verb|{\small Маленький}| текст. \\%1 способ
Какой-нибудь \Large большой \normalsize и нормальный текст. \\%2 способ
\begin{tiny}%3 способ
Малюсенький текст
\end{tiny}



\subsection{Начертание шрифта}

\begin{itemize}
\item семейство: означает примерно (но не в точности) то же, что отечественный термин «гарнитура». Это: rmfamily (шрифты с засечками), sffamily (шрифты без засечек) и ttfamily (шрифты типа «пишущая машинка»); 
\item насыщенность: определяет ширину и жирность шрифта. Возможны насыщенности: средняя(mdseries) и полужирная (bfseries);
\item начертание: бывает прямое (upshape), курсивное (itshape), наклонное (slshape) и «капитель» (scshape).
\end{itemize}

\noindent Некоторым сочетаниям атрибутов никакого шрифта не соответствует. В этом случае затребованный, но отсутствующий шрифт заменяется на другой (по возможности, с близкими атрибутами).

\begin{table}[!h]
\caption{Начертание шрифта}
\centering
\begin{tabular}{|c|c|c|}
 \hline Без аргументов & С аргументом(+) & На печати выйдет \\
\hline \verb|{\rmfamily Шрифт}| & \verb|\textrm{Шрифт}| & \textrm{Шрифт} \\
\hline \verb|{\sffamily Шрифт}| & \verb|\textsf{Шрифт}| & \textsf{Шрифт} \\
\hline \verb|{\ttfamily Шрифт}| & \verb|\texttt{Шрифт}| & \texttt{Шрифт} \\
\hline \verb|{\mdseries Шрифт}| & \verb|\textmd{Шрифт}| & \textmd{Шрифт} \\
\hline \verb|{\bfseries Шрифт}| & \verb|\textbf{Шрифт}| & \textbf{Шрифт} \\
\hline \verb|{\upshape Шрифт }| & \verb|\textup{Шрифт}| & \textup{Шрифт} \\
\hline \verb|{\itshape Шрифт }| & \verb|\textit{Шрифт}| & \textit{Шрифт} \\
\hline \verb|{\slshape Шрифт }| & \verb|\textsl{Шрифт}| & \textsl{Шрифт} \\
\hline \verb|{\scshape Шрифт }| & \verb|\textsc{Шрифт}| & \textsc{Шрифт} \\
\hline
\end{tabular}
\end{table}

\noindent\textbf{Жирный} \textit{курсивный} текст \\
\verb|\normalfont или \textnormal| возвращает обычный шрифт \\
\verb|\renewcommand{\bfdefault}{b}|- уменьшает расстояние между буквами в полужирном шрифте(для красоты) \\
\begin{itemize}
\item[] \textbf{Полужирный шрифт до}
\renewcommand{\bfdefault}{b}
\item[] \textbf{Полужирный шрифт после}
\end{itemize}
Умное выделение - \verb|\emph{Выд \emph{елен} ный}| текст :
\emph{Выд \emph{елен} ный} текст \\
\verb|\textcolor{red}{красный цвет}| - \textcolor{red}{Текст красного цвета}



\subsection{Выравнивание текста}

Окружения для выравнивания текста:
\begin{itemize}
\item \verb|\begin{quote}|- окружение цитат
\item \verb|\begin{center}|-выравн. по центру
\item \verb|\begin{flushleft}|- по левому
\item \verb|\begin{flushright}|- по правому краю
\end{itemize}

\noindent Чтобы показать сам код \LaTeX а:
\verb|\verb"\dots"|-(вместо " может быть любой символ, не встречающийся в коде), или окружение:
\begin{verbatim}
\begin{verbatim}
код...
end{verbatim}
\end{verbatim}

Так же пакет \verb|\usepackage{moreverb}|: подключает окружение listing с буквальным воспроизведением кода, с отступами и номерами строк- \\
\verb|\begin{listing}[1-шаг]{1-началь. номер}| \\
\verb|\listinginput[1]{1}{Analyt1.txt}|-берёт данные из внешнего файла



\subsection{Перечни}

\begin{verbatim}
ненумерованый список:
\begin{itemize}
\item пункт
\end{itemize}

нумерованый список:
\begin{enumerate}
\item пункт
\end{enumerate}

перечень с заголовком(словарь):
\begin{description}
\item[термин] определение
\end{description}
\end{verbatim}

\noindent Вложенность списков:
\begin{itemize}
\item Первый пункт
\item[*] Второй пункт - перечень: \begin{enumerate}
\item[$\surd$] Дочерний пункт - перечень: \begin{enumerate}
\item Первый пункт нумерованного списка - перечень: \begin{itemize}
\item Пункт четвертого уровня вложенности
\end{itemize}
\item Второй пункт нумерованного списка
\end{enumerate}
\item Еще дочерний пункт 
\end{enumerate}
\item Третий пункт
\end{itemize}

\noindent Ссылки на пункты списка:
\begin{enumerate}
\item Переходите улицу только на зеленый свет. \label{green}
\item Стоящий трамвай обходить можно, а автобус нет. \label{tram}
\end{enumerate}
Согласно правилу~\ref{tram}, на стр.\pageref{tram}...

\begin{description}
\item[\textit{Черника}] Черные ягоды.
\item[Голубика] Голубые ягоды.%\textbf- по умолч.
\end{description}



\subsection{Пакет Babel - русификация}

\noindent\verb|\renewcommand{\theenumi}{\Asbuk{enumi}}| - переопределение нумерованного списка, вместо цифр - буквы
\begin{enumerate}
\renewcommand{\theenumi}{\Asbuk{enumi}}
\item One
\item Two
\item Three
\end{enumerate}



\subsection{Тонкости набора текста}

\verb|\dots(или \ldots) / \cdots|-многоточие снизу/посередине строки : $\ldots\cdots$\\
\verb|\symbol{104}|-  ввод любого символа с ASCII-кодом : \symbol{104}\\
Подсказка «АВИКОСУЯ»- все русские однобуквенные слова\\
\verb|\marginpar[$\Longrightarrow$]{$\Longleftarrow$}|-заметки на полях, если справа(необязательный аргумент), слева(обязателный) : \marginpar[$\rightarrow$]{$\leftarrow$}\\
\verb|\righthyphenmin=2| -переносы с отрывом 2 последних букв будут разрешены, как в рус.яз.(3 буквы - в англ.яз.), пример: по-ле \\
\verb|\hyphenation{вклю-чен об-ласть ар-буз}|- указание переносов в нескольких словах, если LaTeX не знает как перенести слово \\
\verb|тво\-рог|- одноразовый способ указания переноса \\
\verb|\tolerance=400|- разреженность строк(200- по умоч., можно 300-400) \\
пистолет-пулемет(дефис \verb|-|) \\
Лето -- это жара(короткое тире \verb|--|) \\
Сказка --- ложь(длинное тире \verb|---|) \\
Москва "--- столица РФ.(между подлежащ. и сказуем. \verb|"---|) \\
"--*Сударь, вы -- полоумный.(прямая речь \verb|"--*|) \\
<<Ёлочки и ,,лапки` `>>(вложенность кавычек \verb|<<>> ,,``|) \\
\verb|\hrulefill{ширина}| - линейка по нижней части строки. \\
\newcommand{\hrf}[1]{\hbox to#1{\hrulefill}}
\verb|\newcommand{\hrf}[1]{\hbox to#1{\hrulefill}}| - для удобства можно создать свою сокращ. команду для линейки \\
Дата в бланке : <<\hrf{2em}>> \hrf{6em} 200\hrf{1em}~г. \\
\verb|<<\hrf{2em}>> \hrf{6em} 200\hrf{1em}~г.|



\subsection{Пробелы}

В русском языке принято:

\begin{enumerate}
\item еденицы изм., процент при переносе не долж. отделяться(неразрывный пробел~): \verb|10~кВт|: 10~кВт, 20~мПа, 30~\%
\item градусы углов: \verb|$45^\circ$|: $45^\circ$     градусы температ.: \verb|23~${}^\circ$C|: 23~${}^\circ$C
\item номер и параграф: \verb|№~3, \S~8'|: №~3, \S~8
\item стандарт. сокращ-я: \verb|т.\:е., и~т.\:д., и~т.\:п.|--- т.\:е., и~т.\:д., и~т.\:п.
\item неразрывать предлоги с послед-м словом: \verb|через~поляну|: на~речку, к~дому, в~течение
\end{enumerate}



\subsection{Колонки}

\begin{verbatim}
\begin{multicols}{3}
Текст в 3 колонки
\end{multicols}
\end{verbatim}

\noindent Параметры многоколоночности(в окружении или преамбуле): \\
\verb|\columnsep=10pt|-расстояние между колонками \\
\verb|\columnseprule=0.4pt| -линейка между колонками(0.4-оптимальный вариант)

\begin{multicols}{2}
\columnsep=20pt
\columnseprule=0.5pt
Проснувшись и взглянув на часы, я увидел, что уже половина седьмого. Ни один из моих людей еще не встал. Чтобы не задерживаться и быстрее отправиться в путь, я решил поднять их всех сразу и выстрелил из двустволки. Они быстро вскочили , а затем, крича стали уверять меня, будто совсем оглохли. Я сказал чтобы они не волновались, попросил скорее развести огонь и готовить завтрак. Выяснив, в чем дело, они мирно возвратились обратно, я же, позавтракав, отправился осматривать окрестности, чтобы выбрать более удобный путь для продолжения нашей прогулки.
\end{multicols}

\begin{enumerate}
\begin{multicols}{3}
\item $2+2=4$
\item $3+3=6$
\item $4+4=8$
\item $5+5=10$
\item $6+6=12$
\item $7+7=14$
\end{multicols}
\end{enumerate}



\subsection{Создание нумерованных спец. объектов - newtheorem}

Команда \verb|\newtheorem| создаёт различные окружения для создания самих теорем(theorem), утверждений(proposition), следствий(corollary), задач(problem), решений(nonum) и т.д., называть их можно по своему, и обычно они задаются в преамбуле: \\

\verb|\newtheorem{theоrem}{Теорема}[section]| \\%1.1
newtheorem - команда создания н-р теоремы \\
theorem - новое окружение для теорем \\
Теорема - название \\
section - подчинение счётчику разделов: будет нумероваться 1.1, 3.4, 5.2

\begin{theorem}[Простое равенство] \label{theorem1}%теорема
Теорема: $2+2=4$
\end{theorem}

\begin{verbatim}
\begin{theorem}[Простое равенство] \label{theorem1}%теорема
Теорема: $2+2=4$
\end{theorem}
\end{verbatim}

Какой-то текст, ещё текст. Смотри теорему \ref{theorem1} на стр. \pageref{theorem1}.

\begin{proposition}%утверждение
Утверждение: $3\times3=9$
\end{proposition}

\begin{corollary} \label{corol1}%следствие
Следствие: $5/5=1$
\end{corollary}

\begin{verbatim}
\begin{corollary} \label{corol1}%следствие
Следствие: $5/5=1$
\end{corollary}
\end{verbatim}

Какой-то текст, ещё текст. Согласно следствию \ref{corol1} на стр. \pageref{corol1}.

\begin{problem}%задача
Задача: $4+4=8$
\end{problem}

\begin{nonum}%решение
Решение: Что-нибудь(ремарка)
\end{nonum}




\section{Формулы}
\subsection{Простые формулы}

\verb|$2+2=4$ : | Внутритекстовая $2+2=4$ формула. \\
\verb|\[3\times3=9\] - выключная формула : | \[3\times3=9\] \\
\verb|$2^{3^2}$ - степень : | $2^{3^2}$ \\
\verb|\frac,\dfrac,\cfrac,\tfrac|-дробь мелкая, крупная, многоэтажная, дробная степень \\
\verb|$\frac12$, $\frac x 2$ - дроби : | $\frac12$, $\dfrac x 2$ \\
\verb|$\cfrac{1}{\sin\alpha-\left(\cfrac{\pi}{2}\right)}$ : | $\cfrac{1}{\sin\alpha-\left(\cfrac{\pi}{2}\right)}$ \\
$\tg\alpha=\dfrac{\sin\alpha}{\cos\alpha}$ \\

\noindent Химические формулы: \\
$C_{16}H_{34} \to C_8H_{18}+C_8H_{16}$ \\
$CH_2=CH_2+O+H_2O \xrightarrow{KMnO_4} HO-CH_2-CH_2-OH$ \\
Фигурные скобки сверху и снизу: $\underbrace{\overbrace{0 1 2 \dots 9}^{10}\rm A B \dots F}_{16}$ : \\
\verb|$\underbrace{\overbrace{0 1 2\dots 9}^{10}\rm A B \dots F}_{16}$| \\
Выделение разрядов числа: \verb|123\,456\,789| - 123\,456\,789 \\
Мелкие символы над и под знаком: $\underset{y=0}{\overset{z=1}{X}}$ : \\
\verb|\underset{низ}{\overset{верх}{X}}|



\subsection{Математические шрифты и стили}

\noindent По умолч. все буквы в формулах - курсивные, если нужно иное: 
\begin{itemize}
\item \verb|\mathbf{}| $\mathbf{bold}$ - полужирный шрифт
\item \verb|\mathsf{}| $\mathsf{serif}$ - без засечек
\item \verb|$\mathcal{ABCDEFGHIJKLM}$| - $\mathcal{ABCDEFGHIJKLM}$-стандартный рукописный шрифт TEXа
\item \verb|$\mathscr{ABCDEFGHIJKLM}$| - $\mathscr{ABCDEFGHIJKLM}$-витиеватый шрифт, доступный при подключении пакета \verb|\usepackage{mathrsfs}|
\item Шрифты, доступные при подключении пакета \verb|\usepackage{amssymb}|: \\
\verb|$\mathbb{ABCDEFGHIJKLM}$| - $\mathbb{ABCDEFGHIJKLM}$-двойной шрифт \\
\verb|$\mathfrak{ABCDEFGHIJKLM0123456789}$| - $\mathfrak{ABCDEFGHIJKLM0123456789}$  -готический
\end{itemize}

Стили написания мат. формул, если нужно использовать не по стандарту: \\
\begin{itemize}
\item Выключной(\verb|$\displaystyle{формула}$|) : $\displaystyle 1\times(2+3)-4=\frac x2$
\item Текстовый(\verb|$\textstyle{формула}$|) : $\textstyle 1\times(2+3)-4=\frac x2$
\item Индексный(\verb|$\scriptstyle{формула}$|) : $\scriptstyle 1\times(2+3)-4=\frac x2$
\item Подиндексный(\verb|$\scriptscriptstyle{формула}$|) : $\scriptscriptstyle 1\times(2+3)-4=\frac x2$
\end{itemize}



\subsection{Сложные формулы}

Для набора важных формул используются разные окружения. При этом формула автоматически получает номер. А при добавлении к нему метки \verb|\label{formula}| или \verb|\label{eq:form}|, на неё затем можно сослаться в тексте. А если нужно самостоятельно пронумеровать формулу : \verb|$2+2=4$ \eqno(1.1)|  \\
\begin{enumerate}
\item equation для записи формулы :
\begin{equation} \label{eq:speed}
V=\dfrac St
\end{equation}
\begin{verbatim}
\begin{equation} \label{eq:speed}
V=\dfrac St
\end{equation}
\end{verbatim}
\item Окружение multline для записи длинной формулы на несколько строк:
\begin{multline}
1+2+3+4+5+6+7+8+9+10+\dots \\ \dots+49+50+51+\dots \\ \dots+95+96+97+98+99+100=5050
\end{multline}
\item align для выравнивания нескольких формул(каждое нумеруется, если какую-либо не нужно нумеровать, после неё используется команда \verb|\notag|):
\begin{align}
2+2&=4 \label{prost} \\
33+33&=66\notag \\%чтобы не нумеровать одно из ур-й
128-37&=91
\end{align}
\end{enumerate}
Уравнение : \eqref{prost} на странице \pageref{prost} \\
Окружение со * , чтобы не присваивать номер
\begin{align*}%не присваив. №
2+2&=4 \\ 
33+33&=66
\end{align*}

Следующие окружения должны быть обернуты в \verb|\[...\]| -переход в математический режим(не нумеруются) или в окружение equation(нумеруются) : \\
\begin{enumerate}
\item aligned - для нумерации нескольких связанных формул или для создания системы уравнений :
\begin{equation}
\begin{aligned}
2+2&=4 \\
33+33&=66
\end{aligned}
\end{equation}
\begin{verbatim}
\begin{equation} \label{eq:sist_ur}
\left\{
\begin{aligned}
4x+72y&=18 \\
7x-y&=250
\end{aligned}
\right.
\end{equation}
\end{verbatim}
\begin{equation} \label{eq:sist_ur}
\left\{
\begin{aligned}
4x+72y&=18 \\
7x-y&=250
\end{aligned}
\right.
\end{equation}
Система равнений - \eqref{eq:sist_ur} на стр. \pageref{eq:sist_ur}.
\item cases - кусочное задание функций :
\[
|x|=\begin{cases}
x, &\text{если } x \ge 0 \\
-x, &\text{если } x < 0
\end{cases}
\]
\item pmatrix - матрица со скобками(), vmatrix-||, bmatrix-[] :
\[
\begin{pmatrix}
a_{11}&a_{12}&a_{13} \\
a_{21}&a_{22}&a_{23} \\
a_{31}&a_{32}&a_{33}
\end{pmatrix}
\]
\end{enumerate}



\subsection{Тонкости набора формул}

\begin{verbatim}
\скобка\matrix- любые скобки у матрицы
\mbox или \text(исп. asmmath)- обычный текст в формуле
\bigl\bigr- самостоятельное определение размера скобок : \\ (\bigl, \Bigl, \biggl, \Biggl)
\binom xy- дробное расположение чисел, но без черты
\xrightarrow[down]{up}- надпись под/над стрелкой
\end{verbatim}

\noindent Русское и английское написание греческих букв в формулах можно переопределить в преамбуле.
\begin{itemize}
\item[англ.] \verb|$ \epsilon \ge \phi$, $\phi \leq \epsilon$, $\kappa \in \emptyset$|
\item $ \epsilon \ge \phi$, $\phi \leq \epsilon$, $\kappa \in \emptyset$
\item[рус.] \verb|$\varepsilon \geqslant \varphi$,| \\\verb|$\varphi \leqslant \varepsilon$, $\varkappa \in \varnothing$|
\item $\varepsilon \geqslant \varphi$, $\varphi \leqslant \varepsilon$, $\varkappa \in \varnothing$
\end{itemize}



\subsection{Коммутативные диаграммы}

Нужно подключить пакет - amscd - коммутативные диаграммы, \verb|\usepackage{amscd}|
\[
\begin{CD}
0 @>>> E^\prime @>f>> E @>g>> E^{\prime\prime}@>>> 0\\ %-@>f>>- текст над стрел.
@.@VVpV@AfAA@VVrV@|\\ %-стрел. вниз-вверх, @.-нет стрел., @|-верт. равно
0 @>>> F^\prime@>f>> F @>g>> F^{\prime\prime}@= 0 %{}- нет знач-я, @= равно, 
\end{CD}
\]
\begin{verbatim}
\[
\begin{CD}
0 @>>> E^\prime @>f>> E @>g>> E^{\prime\prime}@>>> 0\\
@.@VVpV@AfAA@VVrV@|\\
0 @>>> F^\prime@>f>> F @>g>> F^{\prime\prime}@= 0
\end{CD}
\]
@>>> -простая стрелка
@>a>> -стрелка с обозначением
@<<b< -обратная стрелка
@VVxV -стрелка вниз
@AyAA -стрелка вверх
@A{A}AA -если буквы совпадают
@= -равно
@| -равно вертикальное
@. -пусто
\end{verbatim}

Пакет для более сложных коммутативных диаграмм: \\
\verb|\usepackage[matrix,arrow,curve]{xy}|
$$ 
\xymatrix{ 
&& M’\ar@{o->}[dl]^e \ar@/_1pc/@{-->}[ddll]_u\\ 
& K\ar[rr]^f \ar[dr]^h && L \ar[ul]_a \ar[dl]_g\\ 
L’\ar@{o->}[ur]_d \ar@/_1pc/@{-->}[rrrr]_v && 
M\ar[rr]^p \ar[ll]_c && K’\ar@{o->}[ul]_b 
} 
$$

\begin{verbatim}
$$ 
\xymatrix{ 
&& M’\ar@{o->}[dl]^e \ar@/_1pc/@{-->}[ddll]_u\\ 
& K\ar[rr]^f \ar[dr]^h && L \ar[ul]_a \ar[dl]_g\\ 
L’\ar@{o->}[ur]_d \ar@/_1pc/@{-->}[rrrr]_v && 
M\ar[rr]^p \ar[ll]_c && K’\ar@{o->}[ul]_b 
} 
$$
\end{verbatim}



\subsection{Еще примеры формул}

Пример: арифметические действия с многочленами : \\
$$
\arraycolsep=0.05em
\begin{array}{rrr@{\,}r|r}
x^2&{}+2x&{}-12&&\,x+5\\
\cline{5-5}
x^2&{}+5x&&&\,x-3\\
\cline{1-2}
&{}-3x&{}-12\\
&{}-3x&{}-15\\
\cline{2-3}
&&3
\end{array}
$$



\section{Картинки и таблицы}
\subsection{Картинки}

Окружения figure и table не рисуют картинку/таблицу, они занимаются только размещением картинки/таблицы на странице. \\
Пути к папке с картинками указывают в преамбуле. \\
Еденицы измерения : 
\begin{itemize}
\item mm, cm, in
\item em/ex ширина буквы M /высота буквы x
\item page-/text- {-width/-height} высота/ширина страницы/текста
\end{itemize}
\begin{verbatim}
\begin{figure}
\includegraphics{znak.jpg} %-файл
\caption{Значок} %-название
\end{figure}
\end{verbatim}
\verb|\includegraphics{znak.jpg}|- вставка рисунка без надписи\\
\verb|\includegraphics[scale=0.3]{logo.pdf}|- изменение размера \\
\includegraphics{znak.jpg}



\subsection{Табулятор(примитивные таблицы)}

\begin{tabbing}
начинаем \=продолжаем \= заканчиваем\kill%дублируем, если это самая длинная строка(она не будет напечатана), для проставления табуляций, следующая строка - в обычном режиме
начало\>середина\>конец\\%\= установить позицию табуляции, \quad/\qquad пробел/длинный пробел
раз\>два\>три\\%перейти к позиции табуляции
раз\> два\> три\\ 
начинаем\>продолжаем\>заканчиваем\\ 
\end{tabbing}
\verb|\=| установить позицию табуляции, \\
\verb|\>| перейти к позиции табуляции



\subsection{Простые таблицы}

Окружение tabular :
\begin{verbatim}
\begin{tabular}{|l|c||r|}
\hline Слагаемое1 & Слагаемое2 & Сумма \\[5pt]
\hline a & b & c \\
\hline x & y & z \\
\hline 111 & 222 & 333 \\
\hline
\end{tabular}
\end{verbatim}

\begin{tabular}{|l||c|r|}
\hline Слагаемое1 & Слагаемое2 & Сумма \\[3pt]
\hline a & b & c \\
\hline x & y & z \\
\hline 111 & 222 & 333 \\
\hline
\end{tabular} \\

\noindent В атрибуте {|l|c|r|}- указывается отделение столбцов верт. линией и выравнивание по левому(default), середине и правому краю. \\
\verb|&| - отделяет ячейки таблицы \\
\verb|\hline|-горизонтальная черта(отделяет строки таблицы) \\
\verb|\\[3pt]|- отделяет заголовок

%\setlength{\extrarowheight}{4mm}
\begin{tabular}{|p{50px}|c|p{3cm}|}
 \hline a & b & c \\
 \hline x & y & z \\
 \hline 111 & $\displaystyle\frac{b}{y}$ & 333 \\[4mm]
 %\setlength{\extrarowheight}{0}
 \hline
\end{tabular} \\

\noindent В атрибуте {|p{50px}|c|p{3cm}|} - можно задать ширину столбца вручную, если внутри ячейки длинное предложение или абзац.

Окружение tabularx используется для того чтобы ширина колонок была одинаковая : \\

\begin{tabularx}{\textwidth}{|X|c|X|}%чтобы ширина колонок была одинаковая
 \hline xfkj kkf kkd dvheuj fdkjhj mdfgnj & kdk ksc & kskkz ksjd ksdgd jkjkxz kjrkjk kjkfsg lkjkdsjf kjfjk kjkjjhk kjlhf jkhsa\\
 \hline
\end{tabularx} \\

Окружение tabulary используется для того чтобы высота колонок была одинаковая : \\

\begin{tabulary}{\textwidth}{|C|J|R|}%чтобы высота колонок была одинаковая
 \hline xfkj kkf kkd dvheuj fdkjhj mdfgnj & kdk ksc & kskkz ksjd ksdgd jkjkxz kjrkjk kjkfsg lkjkdsjf kjfjk kjkjjhk kjlhf jkhsa \\
 \hline
\end{tabulary} \\

\begin{tabular}{lr@{--}l@{\qquad Обед\quad}r@{--}l}
Понедельник & $8^{30}$ & 15 & 11 & 12 \\ 
Вторник& 12 & 19 & 15 & 16 \\ 
Среда& 10 & 17 & $12^{30}$ & $13^{15}$ \\ 
Четверг& 9 & 17 & 12 & 13 \\ 
Пятница& 11 & 16 & &\\ 
Суббота& 8 & 14 & 11 & 12\\ 
\end{tabular} \\

\noindent \verb|@{}|- встраивает нужные символы или команды между столбцами : \\
\verb|\begin{tabular}{lr@{--}l@{\qquad Обед\quad}r@{--}l}| \\
\verb|!{\hspace{2pt}}|- использование команд TeX \\
\verb|>{} <{}|-команда до и команда после столбца : \\
\verb|\begin{tabular}{l>{$}l<{$}}%>{}<{}| \\

Таблица арифметических действий над числами : \\

\begin{tabular}{l>{$}l<{$}}%>{}<{}-команда до и после
Сложение & 2+2=4 \\
Вычитание & 3-1=2 \\
Умножение & 4\times4=16 \\
Деление & 8/2=4 \\
\end{tabular} \\

Подключенный стилевой пакет hhline позволяет более тонко настроить линии между столбцами и строками. \\

\begin{tabular}{|c|cc|c|} 
\hline А & Б & В & Г\\ 
\hhline{|=|~~|-|}%тонкая настройка линии
Д & Е & Ж & З\\ \hline 
\end{tabular} \\

\begin{tabular}{||cc||cc||}
\hhline{|t:==:t:==:t|}1 &2 &3 &4\\
5 &6 &7 &8\\
\hhline{#==::==||}А & Б & В & Г\\
Д & Е & Ж & З\\
\hhline{|b:==:b:==:b|}
\end{tabular}



\subsection{Плавающие картинки и таблицы}

Ссылка на таблицу: \ref{tab:mytab} \\
Окружение table подбирает расположение таблицы и упоминания в тексте о ней.
\begin{verbatim}
\begin{table}[!h]
\begin{center}
\caption[Таблица]{Таблица с кучей фишек.}\label{tab:mytab}
\begin{tabular}{|c|c||l||c|c|}
\hline 1 & 2 & 3 & 4 & 5 \\
\hline Первый & Второй & \multicolumn{3}{c|}{Третий -- пятый}\\
\cline{1-1} \cline{3-5} 1 & 2 & 3 & 4 & 5\\
\hline 1 & 2 & 3 & 4 & 5\\
\hline \multirow{3}{*}{Три строки} & 2 & 3 & 4 & 5\\
\cline{2-5} & 2 & 3 & 4 & 5\\
\cline{2-5} & 2 & 3 & 4 & 5\\
\hline
\end{tabular}
\end{center}
\end{table}
\end{verbatim}

\begin{table}[!h]%подбирает расположение таблицы и упоминания в тексте о ней (h- здесь, !h- строго здесь)
\begin{center}%выравн. по центру страницы
\caption[Таблица]{Название таблицы.}\label{tab:mytab}
\begin{tabular}{|c|c|l|c|c|}
\hline 1 & 2 & 3 & 4 & 5 \\
\hline Первый & Второй & \multicolumn{3}{c|}{Третий -- пятый}\\
\cline{1-1} \cline{3-5} 1 & 2 & 3 & 4 & 5\\
\hline 1 & 2 & 3 & 4 & 5\\
\hline \multirow{3}{*}{Три строки} & 2 & 3 & 4 & 5\\
\cline{2-5} & 2 & 3 & 4 & 5\\
\cline{2-5} & 2 & 3 & 4 & 5\\
\hline
\end{tabular}
\end{center}
%\caption{Заголовок мог быть и здесь}
\end{table}

Необязательные аргументы окружения table:
\begin{itemize}
\item t - разместить в верхней части страницы
\item b - разместить в нижней части страницы
\item p - разместить на отдельной странице, целиком состоящей из «плавающих» картинок/таблиц
\item h - разместить прямо там, где она встретилась в исходном тексте
\end{itemize}
[tbp]-по умолчанию, использование <<!>> c аргументом повышает шанс размещения таблицы где нужно(!h) \\
multicolumn и multirow объединяют несколько колонок, строк : \\
\verb/\multicolumn{3-гориз.ячейки}{|c|}{текст на этом месте}/ \\
\verb|\multirow{2-верт.ячейки}{*}{текст на этом месте}| \\
\verb|\cline| - рисует прерывистую горизонтальную линию(если она не на всю ширину таблицы) \\

Окружение longtable создаёт таблицу на нескольких страницах:

\begin{verbatim}
\begin{longtable}{|c|c|c|c|}
\caption{Заголовок большой таблицы.}\\
\hline RND1 & RND2 & RND3 & RND4 \\
\hline \endfirsthead                  - конец главного заголовка
\hline 1 & 2 & 3 & 4 \\
\hline \endhead       -конец заголовка на кажд. промежут. стр-це
\hline
\multicolumn{4}{r}{продолжение следует\ldots} \\
\endfoot               - низ таблицы на кажд. промежут. странице
\hline
\endlastfoot%                          - завершение всей таблицы
0,576745371 & 0,435853468 & 0,36384912 & 0,299047979 \\
...
0,094723947 & 0,091199224 & 0,841117852 & 0,617394243 \\
\end{longtable}
\end{verbatim}

\begin{longtable}{|c|c|c|c|}
\caption{Заголовок большой таблицы.}\\
\hline RND1 & RND2 & RND3 & RND4 \\
\hline \endfirsthead%конец глав. заголовка
\hline 1 & 2 & 3 & 4 \\
\hline \endhead%конец заголовка на кажд. промежут. стр.
\hline
\multicolumn{4}{r}{продолжение следует\ldots} \\
\endfoot%низ табл. на кажд. стр.
\hline
\endlastfoot%завершен. всей табл.
0,576745371 & 0,435853468 & 0,36384912 & 0,299047979 \\
0,064795364 & 0,028454613 & 0,751312059 & 0,693972684 \\
0,263563971 & 0,367508634 & 0,075536384 & 0,337780707 \\
0,957583964 & 0,431948588 & 0,938522377 & 0,464307785 \\
0,815740484 & 0,123129806 & 0,883432767 & 0,760983283 \\
0,445062335 & 0,157424268 & 0,883442259 & 0,300596338 \\
0,187159669 & 0,728663343 & 0,637199982 & 0,765684528 \\
0,41009848 & 0,457031472 & 0,142858106 & 0,602946607 \\
0,43315663 & 0,26058316 & 0,611667007 & 0,400328185 \\
0,824086963 & 0,27304335 & 0,244565296 & 0,219675484 \\
0,109578811 & 0,278478018 & 0,242519359 & 0,414669471 \\
0,220778432 & 0,938106645 & 0,502630894 & 0,910760406 \\
0,905239004 & 0,017835419 & 0,429423867 & 0,299079986 \\
0,604679988 & 0,784786124 & 0,86825382 & 0,003631105 \\
0,725883239 & 0,273875543 & 0,843605984 & 0,607743466 \\
0,555736787 & 0,019487901 & 0,342950631 & 0,537183422 \\
0,309374962 & 0,44331087 & 0,749656403 & 0,966836051 \\
0,274332831 & 0,740197878 & 0,865450742 & 0,792816484 \\
0,968626843 & 0,580215733 & 0,706427331 & 0,879562225 \\
0,281344607 & 0,51362826 & 0,7998827 & 0,270290356 \\
0,885143961 & 0,989455756 & 0,235591368 & 0,693434397 \\
0,505067377 & 0,127308502 & 0,614625825 & 0,277375342 \\
0,663594497 & 0,023550761 & 0,670822594 & 0,302446663 \\
0,094723947 & 0,091199224 & 0,841117852 & 0,617394243 \\
0,490246305 & 0,761569651 & 0,973576975 & 0,51597127 \\
0,631301873 & 0,155944248 & 0,319958965 & 0,198643097 \\
0,853761692 & 0,993889567 & 0,105045533 & 0,837805396 \\
0,149834425 & 0,316419619 & 0,387770251 & 0,552013475 \\
0,269182006 & 0,721020214 & 0,484218147 & 0,552132834 \\
0,668632873 & 0,699511389 & 0,278877959 & 0,021775345 \\
0,62638369 & 0,737702261 & 0,696351048 & 0,256427487 \\
0,922563692 & 0,629514529 & 0,789891184 & 0,019748079 \\
0,366649518 & 0,882085214 & 0,805771543 & 0,461659364 \\
0,178967822 & 0,400706498 & 0,313063544 & 0,425676173 \\
0,328582166 & 0,124008134 & 0,177734655 & 0,653821253 \\
0,555736787 & 0,019487901 & 0,342950631 & 0,537183422 \\
0,309374962 & 0,44331087 & 0,749656403 & 0,966836051 \\
0,274332831 & 0,740197878 & 0,865450742 & 0,792816484 \\
0,968626843 & 0,580215733 & 0,706427331 & 0,879562225 \\
0,281344607 & 0,51362826 & 0,7998827 & 0,270290356 \\
0,885143961 & 0,989455756 & 0,235591368 & 0,693434397 \\
0,505067377 & 0,127308502 & 0,614625825 & 0,277375342 \\
0,663594497 & 0,023550761 & 0,670822594 & 0,302446663 \\
0,094723947 & 0,091199224 & 0,841117852 & 0,617394243 \\
\end{longtable}



\subsection{Обтекание рисунков и таблиц текстом}

\verb|\begin{wrapfigure}{сторона}{ширина или длина}|- l- обтекание слева, r- справа (o-внутренняя сторона, i-внешняя сторона в book) \\
\verb|\begin{wrapfigure}[14]{r}{60pt}|-14-количество обтекающих строк \\

\begin{wrapfigure}[15]{l}{0.3\textwidth}
\includegraphics[width=\linewidth]{logo.pdf}
\caption{Картинка с обтеканием}
\end{wrapfigure}

РИСУНОК dolore magna aliqua. Ut enim ad minim veniam,
quis nostrud exercitation ullamco laboris nisi ut 
aliquip ex ea commodo consequat. Duis aute irure dolor
in reprehenderit in voluptate velit esse
cillum dolore eu fugiat nulla pariatur.

Excepteur sint occaecat cupidatat non proident, sunt in culpa qui 
officia deserunt mollit anim id est laborum.
dolore magna aliqua. Ut enim ad minim veniam,
quis nostrud exercitation ullamco laboris nisi ut 
aliquip ex ea commodo consequat.

Duis aute irure dolor in reprehenderit in voluptate velit esse
cillum dolore eu fugiat nulla pariatur. Excepteur sint
occaecat cupidatat non proident, sunt in culpa qui.

\begin{wraptable}{r}{0.25\linewidth}
\caption{Решетчатая}
\begin{tabular}{|c|c|c|}
 \hline 1 & 2 & 3 \\
\hline a & b & c \\
\hline z & x & y \\
\hline
\end{tabular}
\end{wraptable}

ТАБЛИЦА Duis aute irure dolor 
in reprehenderit in voluptate velit esse
cillum dolore eu fugiat nulla pariatur. Excepteur sint
occaecat cupid dolore magna aliqua. Duis aute irure dolor
in reprehenderit.

Ut enim ad minim veniam, quis nostrud exercitation ullamco
 laboris nisi ut.
aliquip ex ea commodo consequat. Duis aute irure dolor
in reprehenderit in voluptate velit esse
cillum dolore eu fugiat nulla pariatur.

Excepteur sint occaecat cupidatat non proident, sunt in culpa qui 
officia deserunt mollit anim id est laborum.
dolore magna aliqua. Ut enim ad minim veniam,
quis nostrud exercitation ullamco laboris nisi ut 
aliquip ex ea commodo consequat.



\section{Инструменты}
\subsection{Свои команды}

Команды задаются глобально(на весь документ - в преамбуле) или локально в каком-нибудь окружении. \\
\verb|\newcommand{\z}{\par\noindent\textbf{Задача.}}|-1)новая команда- создание и оформление задачи. \\
\verb|\newcommand{\str}[1]{на странице \pageref{#1}}|- ссылка на странице... \\
\verb|\newcommand{\smb}[2]{\left(\frac{#1}{#2}\right)}|- команда с 2 аргументами \\
Команда с 2 аргументами, один из них необязательный(x-по умолчанию):
\begin{verbatim}
\newcommand{\qwerty}[2][x]{
\begin{equation}
#1 \nw #2
\end{equation}
}
\end{verbatim}

\verb|\renewcommand{\chaptername}{Глава}|-переопределение команды(использовать с осторожностью).

$x \ge y$%до
\verb| до \renewcommand{\ge}{\geqslant} после |
\renewcommand{\ge}{\geqslant}
$x \ge y$%после

\verb|\newcommand* и \renewcommand*| используется чаще для выявления ошибки, если в тексте команды встретится незапланированная пустая строка(\verb|\\ или \par|), например, если пропущена скобка. поэтому на пустой строке оставляют знак \%



\subsection{Счетчики встроенные и созданные}

\begin{verbatim}
\newcounter{abcd} - новый счетчик=0
\setcounter{abcd}{100} - установить счетчик в 100=100
\addtocounter{abcd}{-27} - добавить к счетчику -27=73
\end{verbatim}

Вывод значения счетчика:
\begin{verbatim}
\arabic{abcd} |арабские
\roman{abcd}  |римские мал.
\Roman{abcd}  |римские боль.
\alph{abcd}   |латиница
\asbuk{abcd}  |русские
\end{verbatim}

\noindent\verb|\arabic{section}| - встроеный счетчик разделов \\
\verb|\renewcommand{\thesection}{Asbuk{section}}| - переопределение встроенного счетчика разделов \\
\verb|\setcounter{section}{0}| - обнуление счетчика разделов \\
\begin{verbatim}
\newcounter{nc} - Команда создание задачи
\newcommand{\z}[1]{
%
\addtocounter{nc}{1} - увеличить счетч. на 1
Задача \thesubsection.\arabic{nc}. #1 - текст задачи
}
\end{verbatim}
\newcounter{nc}[section]
\newcommand{\z}[1]{%

\addtocounter{nc}{1}%
Задача \thesubsection.\arabic{nc}. #1%
}

\z{Текст задачи}

\z{Текст другой задачи}

\z{Найти...}\label{prove}

\noindent В задаче \ref{prove} предлагалось найти...

%\renewcommand{\z}{\thesection.\arabic{\z}} - добавить в преамбулу, тогда в примере выше: в задаче 1.1...



\subsection{Свои окружения}

Для создания задачника, можно так же использовать не новую команду, а целое окружение :
\begin{verbatim}
\newenvironment{zadacha}[1]{
%
\addtocounter{nc}{1}%
Задача \thesubsection.\arabic{nc}.<<#1>>%
}{\vspace{1cm}}
\end{verbatim}
\newenvironment{zadacha}[1]{%
\addtocounter{nc}{1}%
Задача \thesubsection.\arabic{nc}.<<#1>>%
}{\vspace{.5cm}}

\begin{zadacha}{Такая-то задача}
Текст задачи
\end{zadacha}

\begin{zadacha}{Такая-то еще задача}\label{prove2}
Текст еще задачи
\end{zadacha} \\
В задаче \ref{prove2} предлагалось найти...



\subsection{Etoolbox - действия с бинарными переменными}

\begin{verbatim}
\newbool{answers} - создать новую бинарную переменную(по умол. false)
\booltrue{answers} - присвоить ей true(например задачник с ответами, показать их)
\renewcommand{\z}[2][]{%
\addtocounter{nc}{1} - увеличить счетч. на 1 
Задача \thesection.\arabic{nc}. #2%
\\
\ifbool{answers}{Ответ. #1}{} - если переменная true, \\ то показать ответ
}
\z[6]{Сколько будет $2+2$?}
\end{verbatim}
\newbool{answers}%созд. нов. бинарн. пер-ю(по умол. false)
%\booltrue{answers}%присвоить ей true(например задачник с ответами,показать их)
\renewcommand{\z}[2][]{%
%
\addtocounter{nc}{1}%увеличить счетч. на 1 
Задача \thesection.\arabic{nc}. #2%
\\
\ifbool{answers}{Ответ. #1}{}%если пер-я true, то показ. отв.
}
\z[6]{Сколько будет $2+2$?}



\subsection{Гиперссылки на внеш. документы}

\verb|\url{http://google.com}| --
\url{http://google.com} \\%1-й спос.
\verb|Сайт \href{http://yandex.ru}{Yandex}| --
Сайт \href{http://yandex.ru}{Yandex}%2-й спос.



\section{Окончание документа}
\subsection{Список таблиц и иллюстраций в конце статьи}

\listoffigures%список картинок

\listoftables%список таблиц



\subsection{Библиография}

\subsubsection{Простая библиография}

Смотри первый источник : \cite{qwerty}
а так же смотри второй источник : \cite{fama} \\

Можно переопределить название списка литературы на другое : \\
\verb|\renewcommand{\refname}{Список источников}| \\%по умолч. "Список литературы" (для article)
\verb|\renewcommand{\bibname}{Литературные источники}|%по умолч. "Литература" (для book, report)

Окружение, включающее в себя библиографию документа : \\
\begin{verbatim}
\begin{thebibliography}{9}
\addcontentsline{toc}{section}{\refname}
\bibitem{qwerty} Пупкин В.М, Труд об истории //Образование. -- №8 1998.
 -- С,18.
\bibitem{fama} Иванов И.И, Труд о фундаментальной физике //Наука и жизнь.
 -- №2 1994. -- С. 23-38.
\end{thebibliography}
\end{verbatim}
\verb|{9}| - примерное количество источников(однозначное или двузначное число) \\
\verb|\addcontentsline{toc}{section}{\refname}| - добавить библиографию в содержание: \\
\verb|{toc}|-куда(tableofcontents), \\ \verb|{section}|-уровень пункта библиографии в содержании, \\
\verb|{\refname}|-заглавие списка(Список литературы) \\
\begin{thebibliography}{9}
\addcontentsline{toc}{section}{\refname}%добавить библиографию в содержание:{toc}-tableofcontents,{section}-уровень пункта библиографии в содержании,{\refname}-заглавие списка(Список литературы)
\bibitem{qwerty} Пупкин В.М, Труд об истории //Образование. -- №8 1998. -- С,18.
\bibitem{fama} Иванов И.И, Труд о фундаментальной физике //Наука и жизнь. -- №2 1994. -- С. 23-38.
\end{thebibliography}



%\subsubsection{Библиография - BIB\TeX\ или BIB\LaTeX}

%\printbibliography%список источников

%Смотри статью: \cite{fama1965behavior} 

%Еще \cite{райзер1987физика}.

%И еще \cite{Dufwenberg2013}.



%\subsection{С О Д Е Р Ж А Н И Е}

\tableofcontents%Cодержание(без библиографии)

%(y1-y2)x+(x2-x1)y+(x1y2-x2y1)=0 вычисление ур-я ф-и по точкам

\end{document} % конец документа