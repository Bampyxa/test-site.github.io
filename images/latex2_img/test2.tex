\documentclass[aspectratio=169,
% t, c, или b - вертикальное выравнивание на слайдах (верх, центр, низ)
%handout, - на слайдах всё сразу, без пауз
%aspectratio=169, - Соотношение сторон(43- по умолч.)
%draft - черновая, быстрая печать
]{beamer}

%%%	ОФОРМЛЕНИЕ:
\usetheme{AnnArbor} % Тема оформления:
%AnnArbor, Antibes, Bergen, Berkeley, Berlin, Boadilla, CambridgeUS, Copenhagen, Darmstadt, Dresden, Frankfurt, Goettingen, Hannover, Ilmenau, JuanLesPins, Luebeck, Madrid, Malmoe, Marburg, Montpellier, PaloAlto, Pittsburgh, Rochester, Singapore, Szeged и Warsaw.

%\usecolortheme{beaver} % Цветовая тема:
%albatross, beetle, crane, dolphin, dove, fly, lily, orchid, rose, seagull, seahorse, sidebartab, structure, whale и wolverine.

%\usefonttheme{serif}%-шрифтовая тема:
%pro-fessionalfonts, structurebold, structureitalicserif и structuresmallcapsserif.

%\useinnertheme{circles}%-тема для списков, теорем и выделения: inmargin, rectangles, rounded.

%\useoutertheme{infolines}%-тема для заголовков и обрамления: miniframes, shadow, sidebar, smoothbars, smoothtree, split и tree.

%\usetheme{HSE}%своя цвет. схема(дополн-е файлы)

\usepackage[T2A]{fontenc}
\usepackage[utf8]{inputenc}
\usepackage[english,russian]{babel}

%% Опред-е своих окруж-й theorem,proof,example, только с русскими названиями
\newtheorem{rtheorem}{Теорема}
\newtheorem{rproof}{Доказательство}
\newtheorem{rexample}{Пример}

%% Работа с графикой
\usepackage{graphicx} % Для вставки рисунков
\graphicspath{{images/}{images2/}}
\usepackage{tikz}
\usepackage{pgfplots}
\usepackage{pgfplotstable}

%\includeonlyframes{ftitle,cont,f1,f2,f3,f4,f5,f6,f7,f8,f9,f10,f11,f12,f13,
%f14,f15,f16,f17,f18,f19,f20,f21}%для быстроты включает только выбранные фреймы(желательно каждому присваивать label=...)
%\includeonlyframes{}



\title{Презентации в Beamer}
\subtitle{Фреймы и слайды}
\author{Гога Маленькян}
\date{\today}



\begin{document}

\begin{frame}[label= ftitle]%Титульный слайд
\maketitle
\end{frame}


\section{Оглавление}

\begin{frame}[label=cont,shrink=20] \label{cont}%первый label, чтобы можно было вкл./не вкл. фрейм в презент.(\includeonlyframes{cont}), второй - чтобы созд. где-нибудь ссыл. на данный фрейм(\hyperlink{cont}{ссылка})
\frametitle{\insertsection}
\tableofcontents%[pausesections, pausesubsections-чтобы содержание раскрывалось постепенно; sections={<2-3>}]
\end{frame}



\section{Презентации в \LaTeX}

\begin{frame}[label=f1]
\frametitle{\insertsection}
Для создания презентаций можно использовать не только всем известный PowerPoint от Microsoft, но и такие пакеты \LaTeX а, как \textit{Beamer, PowerDot, Prosper, PPower4, PdfScreen, PdfSlide} и др. \\
Beamer, можно сказать, самый продвинутый из всех, но при этом, простой.
\end{frame}



%\section{Немного о преамбуле}
\subsection{Класс документа, оформление}

\begin{frame}[fragile, label=f2]%fragile-для окр.verbatim; label-для манипуляций с фреймом.
\frametitle{\insertsection}
\framesubtitle{\insertsubsection}

\verb|\documentclass[param]{beamer}| - \\
\textbf{beamer} - класс документа: презентация \\
\textbf{param} - необязательные параметры, через запятую: \\
\textit{t, c,  b} - вертикальное выравнивание на слайдах (верх, центр(default), низ);\\
\textit{aspectratio=169} - Соотношение сторон-16:9(43- по умолч.4:3); \\ \medskip
\verb|\usetheme{CambridgeUS}| - \\
- общая тема оформления, практически у каждого университета своя: \\
\textit{Bergen, Madrid, Szeged, AnnArbor, Pittsburg, Rochester, Antiles, Montpellier, Berkley.}
\end{frame}


\subsection{Оформление, остальные пакеты}

\begin{frame}[fragile, label=f3]
\frametitle{\insertsection}
\framesubtitle{\insertsubsection}

Примеры остальных тем, меняющих только некоторые стороны оформления документа:
\begin{verbatim}
\usecolortheme{beaver}- цветовая тема
\usefonttheme{serif}-шрифтовая тема
\useinnertheme{circles}-тема для списков, теорем и выделения
\useoutertheme{infolines}-тема для заголовков и обрамления
\end{verbatim}
Дальше подключаем те пакеты, которые нам нужны для содержания самой презентации:
текстовые, математические, графические и др.
\end{frame}



\subsection{Фреймы и слайды}

\begin{frame}[label=f4]
\frametitle{\insertsection}
\framesubtitle{\insertsubsection}

Вся презентация делится на \textbf{фреймы}(аналог страницы в обычном документе), в которых помещается вся полезная информация. \\
Эти фреймы могут подразделяться на \textbf{слайды}(оверлеи, представления). Т.е. можно не выводить всю информацию на экран целиком, \pause а делать это \pause  маленькими \pause порциями, \pause \only<5>{а можно что-то убирать.}  \pause При этом, виден будет всё тот же фрейм. \pause \\
Все операции со слайдами делаются внутри окружения \textbf{frame}. Существуют несколько команд для разделения фрейма на слайды. \\
\label<7>{f4_7} \hyperlink{f16_3}{(Ссылка назад)}
\end{frame}



\section{Создание слайдов}
\subsection{Команда pause}

\begin{frame}[fragile, label=f5]
\frametitle{\insertsection}
\framesubtitle{\insertsubsection}

\only<1>{Первая команда - \textbf{pause}, ставится она на границе слайдов, для их разделения:}
\begin{verbatim}
%\begin{frame}{Создание слайдов}
Beamer --- это удобный пакет для создания презентаций. \\ \pause 
Это второй слайд \\ \pause 
Это наш третий слайд \\ \pause 
Четвёртый! Паузу можно поставить в любом \pause месте - это пятый.
%\end{frame}
\end{verbatim} \pause

Beamer ---  это удобный пакет для создания презентаций. \\ \pause
Это второй слайд \\ \pause
Это наш третий слайд \\ \pause
Четвёртый! Паузу можно поставить в любом \pause месте - это пятый.
\end{frame}



\subsection{Постепенный вывод списка(pause)}

\begin{frame}[fragile, label=f6]
\frametitle{\insertsection}
\framesubtitle{\insertsubsection}

\only<1>{Со списком можно проделать то же самое.}
\begin{verbatim}
\begin{itemize}
\item Пункт списка №1: \pause
\begin{itemize}
\item Вложенный пункт а \pause
\item Вложенный пункт б \pause
\end{itemize}
\item Пункт списка №2
\end{itemize}
\end{verbatim} \pause
\begin{itemize}
\item Пункт списка №1: \pause
\begin{itemize}
\item Вложенный пункт а \pause
\item Вложенный пункт б \pause
\end{itemize}
\item Пункт списка №2
\end{itemize}
\end{frame}



\subsection{Команда only}

\begin{frame}[fragile, label=f7]
\frametitle{\insertsection}
\framesubtitle{\insertsubsection}
\only<1>{Существует ещё команда \textbf{only<number-slide>}, которая позволит показать какую-либо информацию, только на определённых слайдах. Например:}
\begin{verbatim}
\only<1,3>{Привет, меня зовут Петя.\\}
\only<2-4>{У меня есть кошка по кличке Нюша.\\}
\only<3>{Мышка по прозванью Валя.\\}
\only<4>{А еще собака -- Барбос.}
\end{verbatim} \pause
\only<2,4>{Привет, меня зовут Петя.\\}
\only<3-5>{У меня есть кошка по кличке Нюша.\\}
\only<4>{Мышка по прозванью Валя.\\}
\only<5>{А еще собака -- Барбос. \\}
\only<6>{Обратите внимание, при использовании этой команды(в отличие от остальных), перенос на новую строку, если он нужен - $\backslash\backslash$, ставится вместе с текстом, внутри скобок.}
\end{frame}



\subsection{Команда uncover}

\begin{frame}[fragile, label=f8]
\frametitle{\insertsection}
%\frametitle{Свое название раздела}
\framesubtitle{\insertsubsection}
%\setbeamercovered{dynamic}%динамическая прозрачность
\only<1>{Следуюшая команда \textbf{uncover}, похожа  на предыдущую, но при этом место под текст или формулу <<резервируется>>.}
\begin{verbatim}
\uncover<1->{За лесами, за морями,}\\
\uncover<3->{За высокими горами.}\\
\uncover<2->{Не на небе - на земле,}\\
\uncover<4->{Жил старик в одном селе.}
\end{verbatim} \pause
\uncover<2->{За лесами, за морями,}\\
\uncover<4->{За высокими горами.}\\
\uncover<3->{Не на небе - на земле,}\\
\uncover<5->{Жил старик в одном селе. \\
\label<5>{f8_5} \hyperlink{f16_3}{(Ссылка назад)}}
\end{frame}



\subsection{Команды visible/invisible}
\begin{frame}[fragile, label=f9]
\frametitle{\insertsection}
\framesubtitle{\insertsubsection}
\only<1>{В принципе, следующие команды - \textbf{visible/invisible}, похожи на uncover, но более наглядны в использовании. Первая занимается отображением на указанных слайдах, вторая наоборот - скрытием на данных слайдах.}
\begin{verbatim}
\visible<1-2>{формула 1:}\invisible<1-2>{$2H_2O\to2H_2+O_2$} \\
\visible<2>{формула 2:}\invisible<1-2>{$\tg\alpha=\sin\alpha/\cos
\alpha$} \\
\visible<3>{}
\end{verbatim} \pause
\visible<2-3>{формула 1:}\invisible<2-3>{$2H_2O\to2H_2+O_2$} \\
\visible<3>{формула 2:}\invisible<2-3>{$\tg\alpha=\sin\alpha/\cos\alpha$} \\
\visible<4>{}
\end{frame}



\subsection{Команды alt/temporal}
\begin{frame}[fragile, label=f10]
\frametitle{\insertsection}
\framesubtitle{\insertsubsection}
\only<1>{Более сложные результаты достигаются с помощью команд  \textbf{alt} и  \textbf{temporal}. Первая позволяет выводить один текст на определенных слайдах и другой на всех остальных, а вторая предлагает целых три варианта: до указанных слайдов, на указанных слайдах и после указанных слайдов.}
\begin{verbatim}
\alt<1>{Это первый - \textbf{нужный} слайд}{Это точно не первый
 слайд}\\
\temporal<2>{Приготовься!!!}{Здесь во 2-м слайде важная информация}
{Расслабься:)} \\
\only<3>{}
\end{verbatim} \pause
\alt<2>{Это первый - \textbf{нужный} слайд}{Это точно не первый слайд} \\
\temporal<3>{Приготовься!!!}{Здесь во 2-м слайде \textbf{важная} информация}{Расслабься:)} \\
\only<4>{}
\end{frame}



\subsection{Окружения для создания слайдов}
\begin{frame}[fragile, label=11]
\frametitle{\insertsection}
\framesubtitle{\insertsubsection}
\only<1>{Допустим нужно разместить некоторый текст, несколько формул или таблицу. Логичнее было бы объединить их в некоторое окружение. Это можно сделать так как, каждой команде для создания оверлея соответствует аналогичное окружение с названием заканчивающимся на  \textbf{-env: onlyenv, uncoverenv} и др.}

\begin{onlyenv}<2>
\begin{verbatim}
\begin{onlyenv}<1>
$2\times2=4$ \\
$3\times3=9$ \\
$4\times4=16$ \\
$5\times5=25$ \\
\end{onlyenv}
\begin{onlyenv}<2>
\includegraphics[scale=0.1]{pi.jpg}
\end{onlyenv}
\end{verbatim}
\end{onlyenv} \pause

\begin{onlyenv}<3-4>
$2\times2=4$ \\
$3\times3=9$ \\
$4\times4=16$ \\
$5\times5=25$ \\
\end{onlyenv}
\begin{onlyenv}<4>
\includegraphics[scale=0.1]{pi.jpg}
\end{onlyenv}
\end{frame}


\section{Общее о презентации}
\subsection{Другие команды}
\begin{frame}[fragile, label=f12]
\frametitle{\insertsection}
\framesubtitle{\insertsubsection}
\only<1>{Содержимое фрейма можно не только скрыть/показать. Но и как-то менять например шрифт, цвет и т.д. Практически все команды поддерживают изменения на разных слайдах.}
\begin{verbatim}
\textbf<1>{Эта строчка жирная только на 1 слайде. \\}
\textit<2>{Эта строка курсивная только на 2 слайде. \\}
\textcolor<3>[RGB]{0,255,0}{Текст зелёный только на 3 слайде.\\}
В этом тексте \alert<4>{данное} слово выделено красным только
 на 4 слайде.
\end{verbatim} \pause
\textbf<2>{Эта строчка жирная только на 1 слайде. } \\
\textit<3>{Эта строка курсивная только на 2 слайде. } \\
\textcolor<4>[RGB]{0,255,0}{Этот текст зелёный только на 3 слайде. } \\
В этом тексте \alert<5>{данное} слово выделено только на 4 слайде.
%\box<3,5>{Этот текст в обрамлении только на слайдах 3 и 5.}
\end{frame}



\subsection{<<Правильное>> использование оверлеев}

\begin{frame}[fragile, label=f13]
\frametitle{\insertsection}
\framesubtitle{\insertsubsection}
\only<1>{Для удобства можно с помощью команды \textbf{newcommand} сокращать запись и если переход от одного пункта к другому не случайный, а по порядку, то можно вместо цифр использовать $+$. При этом счётчик сам будет увеличиваться на 1, а вы сможете добавлять новые пункты списка, не сбивая нумерацию.}
\begin{onlyenv}<2>
\begin{verbatim}
\newcommand{\green}{\only{\textcolor[RGB]{128,128,0}}}
\begin{itemize}
\item \green<+>{Первый пункт}
\item \green<+>{Полуторный пункт}
\item \green<+>{Второй пункт}
\item \green<+>{Третий пункт}
\end{itemize}
\end{verbatim}
\end{onlyenv}

\begin{onlyenv}<3->
\newcommand{\green}{\only{\textcolor[RGB]{128,128,0}}}
\begin{itemize}
\item \green<3>{Первый пункт}
\item \green<4>{Полуторный пункт}
\item \green<5>{Второй пункт}
\item \green<6>{Третий пункт}
\end{itemize}
\end{onlyenv}
\end{frame}



\subsection{Логические блоки}

\begin{frame}[fragile, label=f14]
\frametitle{\insertsection}
\framesubtitle{\insertsubsection}
\only<1>{В презентациях можно подразделять текст на некоторые логические блоки, с помощью окружения \textbf{block}. Как они будут выглядеть, зависит от темы.}
\begin{onlyenv}<2>
\begin{verbatim}
\begin{block}{Первый блок}
Некоторое содержимое данного блока. \\
Ещё добавим и это предложение.
\end{block}
\begin{block}{Второй блок}
$1+2+3+4+5+\dots+96+97+98+99+100=5050$
\end{block}
\begin{block}{Третий блок}
В данном блоке находится кнопка-ссылка на другой фрейм. \\
При клике на ней мы перенесёмся туда. \\
\hyperlink{lab}{\beamerbutton{Кнопка со ссылкой}}
\end{block}
\end{verbatim}
\end{onlyenv}

\begin{onlyenv}<3>
\begin{block}{Первый блок}%отдел-ся некот. образом от остального содерж.
Некоторое содержимое данного блока. \\
Ещё добавим и это предложение.
\end{block}
\begin{block}{Второй блок}
$1+2+3+4+5+\dots+96+97+98+99+100=5050$
\end{block}
\begin{block}{Третий блок}
В данном блоке находится кнопка-ссылка на другой фрейм. \\
При клике на ней мы перенесёмся туда. \\
\hyperlink{big_text}{\beamerbutton{Кнопка со ссылкой}}% ведет на 2стр. вперед
\end{block}
\end{onlyenv}
\end{frame}



\subsection{Теоремы}

\begin{frame}[fragile, label=f15]
\frametitle{\insertsection}
\framesubtitle{\insertsubsection}
\begin{onlyenv}<1>
Разделять некоторый текст по блокам можно и с помощью окружений \textbf{theorem, proof, example}. Это даже лучше, так как они специально созданы для этого. Но так как их названия не русифицированы, то лучше создать свои окружения в преамбуле: \\
\begin{verbatim}
\newtheorem{rtheorem}{Теорема}
\newtheorem{rproof}{Доказательство}
\newtheorem{rexample}{Пример}
\end{verbatim}
\end{onlyenv}

\begin{onlyenv}<2>
\begin{verbatim}
\begin{rtheorem}
В равностороннем треугольнике $\triangle ABC$ все углы тоже
 равны. (Пифагор)
\end{rtheorem}
\begin{rproof}
Пифагоровы штаны на все стороны равны.
\end{rproof}
\begin{rexample}
$AB=BC, BC=AC, AB=AC : \angle A=\angle B=\angle C$
\end{rexample}
\end{verbatim}
\end{onlyenv}

\begin{onlyenv}<3>
\begin{rtheorem}
В равностороннем треугольнике $\triangle ABC$ все углы тоже равны. (Пифагор)
\end{rtheorem}
\begin{rproof}
Пифагоровы штаны на все стороны равны.
\end{rproof}
\begin{rexample}
$AB=BC, BC=AC, AB=AC : \angle A=\angle B=\angle C$
\end{rexample}
\end{onlyenv}
\end{frame}



\subsection{Ссылки в презентациях}

\begin{frame}[fragile, label=f16] %\hypertarget<3>{back}{}
%\hypertarget<№ оверлея>{proof_theor_1}{Смотри здесь}, \hypertarget<5>{backw_slide}{\hyperlink{forw_slide}{Ссылка назад}} - более общий метод указания метки
\label<3>{f16_3}
\frametitle{\insertsection}
\framesubtitle{\insertsubsection}
\only<1>{Ссылки в презентациях как правило перекрёстные. Например перенестись к уже пролистаному фрейму или забежать немного вперёд, а потом в один клик сразу вернуться назад и продолжить презентацию.} \pause
\begin{onlyenv}<2>
\begin{verbatim}
\ begin{frame}[label=метка] -
 - это метка фрейма - адрес, куда перенесёт нас ссылка.
\label<№слайда>{метка} -
 - а эту метка для указания на определённый слайд фрейма
\hyperlink{theor_1}{Ссылка} -
 - это сама ссылка где-либо в тексте, в ней указана метка и текст
 самой ссылки.
\beamerbutton{Кнопка со ссылкой},\beamergotobutton{значок-стрелочка},
\beamerskipbutton{двойная стрелочка},\beamerreturnbutton{стрелка влево} - 
 - вместо слова `Ссылка` может быть такая стилизованная кнопка.
\label<7>{f4_7} \hyperlink{f16_3}{(Ссылка на 7-й слайд 4 фрейма)},
\label<3>{f16_3} \hyperlink{f4_7}{(Ссылка назад, на 3-й слайд 16 фрейма)} -
 - так будут выглядеть две перекрёстные ссылки.
\end{verbatim}
\end{onlyenv}
\only<3>{
\hyperlink{big_text}{\beamerbutton{Ссылка на следующий фрейм}} \\
\hyperlink{f8_5}{Ссылка на слайд далеко позади} \\
\hyperlink{f4_7}{Ссылка на 4-й слайд}
}
\end{frame}


\subsection{Дополнительные параметры frame}
\begin{frame}[containsverbatim, fragile, label=f17]
Каждое окружение frame имеет следующие необязательные опции, \\
\verb|\ begin{frame}[param]| : \\
(которые, как вы заметили можно указать и в преамбуле(при указании класса), для всех фреймов)
\begin{itemize}
\item containsverbatim — если есть окружение verbatim или команда \verb|\verb| для отображения кода;
\item label=метка — метка слайда (для команды \verb|\ref| и гиперссылок);
\item b, c, t — выравнивание (по верхнему краю, по центру(default), по нижнему краю;
\item plain — слайд без оформления;
\item shrink=число — ужимает содержимое кадра на «число» процентов;
\item squeeze — другой способ, уменьшает вертикальные промежутки;
\item fragile — говорит beamer-у, что внутри кадра размещается «хрупкий» текст (н-р, verbatim)
\end{itemize}
\end{frame}
\frametitle{\insertsection}
\framesubtitle{\insertsubsection}


\subsection{Большой текст}

\begin{frame}[shrink=10, fragile, label=f18] \label{big_text}%shrink- ужать текст на 7%(лучше не использовать)
\frametitle{\insertsection}
\framesubtitle{\insertsubsection}
Если по какой-то причине текст немного не умещается в слайде(одна-две строки), используйте \\
\verb|\ begin{frame}[shrink=10]| \\
И содержимое ужмётся. Но злоупотреблять этим не стоит, лучше сделать два слайда или фрейма. \\ \medskip
\textsl{Чуден Днепр при тихой погоде, когда вольно и плавно мчит сквозь леса и горы полные воды свои. Ни зашелохнет; ни прогремит. Глядишь, и не знаешь, идет или не идет его величавая ширина, и чудится, будто весь вылит он из стекла, и будто голубая зеркальная дорога, без меры в ширину, без конца в длину, реет и вьется по зеленому миру. Любо тогда и жаркому солнцу оглядеться с вышины и погрузить лучи в холод стеклянных вод и прибережным лесам ярко отсветиться в водах. Зеленокудрые! они толпятся вместе с полевыми цветами к водам и, наклонившись, глядят в них и не наглядятся, и не налюбуются светлым  своим зраком, и усмехаются к нему, и приветствуют его, кивая ветвями. В середину же Днепра они не смеют глянуть: никто, кроме солнца и голубого неба, не глядит в него. Редкая птица долетит до середины Днепра. Пышный! ему нет равной реки в мире. Чуден Днепр и при теплой летней ночи, когда все засыпает --- и человек, и зверь, и птица; а бог один величаво озирает небо и землю и величаво сотрясает ризу.} \\
\hfill{\textit{Н. В. Гоголь}}
\end{frame}



\begin{frame}[fragile, label=f19]{Пошаговое построение графика}%можно написать название и здесь
\only<1>{Постепенно добавлять можно не только текст, формулы, но и простые рисунки, графики, диаграммы, созданные с помощью графических пакетов, о которых мы поговорим в следующей части.}
\begin{onlyenv}<2>
\begin{verbatim}
\begin{tikzpicture}
\draw<1->[->, ultra thick] (0,0) --(0,5.2) node[left]{y};
\draw<2->[->, ultra thick] (0,0)--(5.2,0) node[below]{x};
\draw<3->[help lines] grid(5,5);
\draw<4-> (0,0)--(5,4);
\draw<5-> (0,5)--(5,1);
\draw<6->[domain=0:2] plot(\x, \x*\x);
\end{tikzpicture}
\end{verbatim}
\end{onlyenv}

\begin{onlyenv}<3->
\begin{tikzpicture}
\draw<3->[->, ultra thick] (0,0)--(0,5.2) node[left]{y};
\draw<4->[->, ultra thick] (0,0)--(5.2,0) node[below]{x};
\draw<5->[help lines] grid(5,5);
\draw<6-> (0,0)--(5,4);
\draw<7-> (0,5)--(5,1);
\draw<8->[domain=0:2] plot(\x, \x*\x);
\end{tikzpicture}
\end{onlyenv}
\end{frame}


\subsection{Манипуляции с фреймами}

\begin{frame}[fragile, label=f20]
\frametitle{\insertsection}
\framesubtitle{\insertsubsection}
Если у вас в презентации есть несколько фреймов. которые не относятся к теме, не обязательно их удалять или закоментировать все строки знаком - \%. \\
Достаточно каждому фрейму прописать необязательный параметр - label : \verb|\ begin{frame}[label=frm1]|. \\
И в преамбуле указать с помощью команды - \verb|\includeonlyframes{frm1,frm2}|, какие из них вы хотите добавить в презентацию, остальные будут игнорироваться. Таким образом всегда можно регулировать содержание.
\end{frame}



\section{<<Лекции>>}

\includeonlylecture{lec1,lec2}%показать только лекцию 1(Можно поместить это в преамбулу)

\lecture{Лекция 1}{lec1}
\subsection{Лекция 1}
\begin{frame}[fragile, label=f21]
\frametitle{\insertsection}
\framesubtitle{\insertsubsection}
У lecture-функционала, такое же назначение, он позволяет включать в презентацию только отдельные слайды, но при этом он более объединяющий. \\
Допустим какие-то фреймы относятся ко всему курсу(титульный лист, оглавление и др.), поэтому размещены в начале файла. А остальные фреймы разделены между тремя лекциями, и идут друг за другом. В начале каждой из них указывается команда - \\
\verb|\lecture{Лекция №1}{lec1}| - с названием и меткой\\
При этом в конце фреймов, не относящихся ни к какой лекции задаётся команда - \\
\verb|\includeonlylecture{lec1,lec2}| - которая показывает какие лекции нужно включить, а какие нет, в презентацию.
\end{frame}

\lecture{Лекция 2}{lec2}
\subsection{Лекция 2}
\begin{frame}[fragile, label=f22]
\frametitle{\insertsection}
\framesubtitle{\insertsubsection}
Ещё раз, чтобы было понятнее. Все frames не внутри какой-нибудь lecture присутствуют всегда. Если нет  команды \verb"includeonlylecture{}", то все лекции тоже присутствуют всегда. Отдельную лекцию можно показать с помощью \verb"includeonlylecture{lec1}". \\
\begin{verbatim}
\includeonlylecture{lec1} - будет показана только лекция 1.
\lecture{Лекция 1}{lec1}
%\begin{frame}{Фрейм №1 Лекция №1}
Какое-то содержимое фрейма.
%\end{frame}
После него еще несколько фреймов...
\lecture{Лекция 2}{lec2}
%\begin{frame}{Фрейм №1 Лекция №2}
Какое-то содержимое фрейма.
%\end{frame}
\end{verbatim}
\end{frame}



\end{document}